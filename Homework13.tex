\documentclass[a4paper]{ctexart}
\usepackage{geometry}
% useful packages.
\usepackage{amsfonts}
\usepackage{amsmath}
\usepackage{amssymb}
\usepackage{amsthm}
\usepackage{mathrsfs}
\usepackage{enumerate}
\usepackage{graphicx}
\usepackage{multicol}
\usepackage{fancyhdr}
\usepackage{layout}
\newtheorem{Definition}{\hspace{2em}定义}
\newtheorem{Example}{\hspace{2em}例}
\newtheorem{Thm}{\hspace{2em}定理}
\newtheorem{Lem}{\hspace{2em}引理}
\newtheorem{cor}{\hspace{2em}推论}
% some common command
\newcommand{\dif}{\mathrm{d}}
\newcommand{\avg}[1]{\left\langle #1 \right\rangle}
\newcommand{\difFrac}[2]{\frac{\dif #1}{\dif #2}}
\newcommand{\pdfFrac}[2]{\frac{\partial #1}{\partial #2}}
\newcommand{\OFL}{\mathrm{OFL}}
\newcommand{\UFL}{\mathrm{UFL}}
\newcommand{\fl}{\mathrm{fl}}
\newcommand{\op}{\odot}
\newcommand{\cp}{\cdot}
\newcommand{\Eabs}{E_{\mathrm{abs}}}
\newcommand{\Erel}{E_{\mathrm{rel}}}
\newcommand{\DR}{\mathcal{D}_{\widetilde{LN}}^{n}}
\newcommand{\add}[2]{\sum_{#1=1}^{#2}}
\newcommand{\innerprod}[2]{\left<#1,#2\right>}
\newcommand\tbbint{{-\mkern -16mu\int}}
\newcommand\tbint{{\mathchar '26\mkern -14mu\int}}
\newcommand\dbbint{{-\mkern -19mu\int}}
\newcommand\dbint{{\mathchar '26\mkern -18mu\int}}
\newcommand\bint{
{\mathchoice{\dbint}{\tbint}{\tbint}{\tbint}}
}
\newcommand\bbint{
{\mathchoice{\dbbint}{\tbbint}{\tbbint}{\tbbint}}
}
\title{Homework\# 13}
\author{Shuang Hu(26)}
\begin{document}
\maketitle
\section*{P124 Problem4}
\begin{proof}
    If $\lambda\in\Lambda$, $\exists u\neq 0$ such that
    \begin{equation}
        \label{eq:CharacterEq}
        \left\{
            \begin{aligned}
                (L-\lambda)u&=0,x\in\Omega\\
                u&=0,x\in\partial\Omega\\
            \end{aligned}
        \right.
    \end{equation}
    In $\Omega$, $Lu=\lambda u$, it means:
    \begin{equation}
        \label{eq:Lutimesu}
        \innerprod{Lu}{u}_{L^{2}(\Omega)}=\innerprod{\lambda u}{u}=\lambda\innerprod{u}{u}.
    \end{equation}
    On the other hand, for $L=L^{*}$, it derives the following equation:
    \begin{equation}
        \label{eq:Lstaru}
        \innerprod{Lu}{u}=\innerprod{u}{L^{*}u}=\innerprod{u}{Lu}=\innerprod{u}{\lambda u}=\bar{\lambda}\innerprod{u}{u}.
    \end{equation}
    As $u\neq 0$, $\innerprod{u}{u}>0$. By \eqref{eq:Lstaru} and \eqref{eq:Lutimesu}, we can see $\lambda=\bar{\lambda}$, which means that $\lambda\in\mathbb{R}$. So $\Lambda\subset\mathbb{R}$.
\end{proof}
\section*{P124 Problem5}
For the eigen-value problem 
\begin{equation}
    \left\{
        \begin{aligned}
            -\Delta u&=\lambda u,x\in\Omega\\
            u&=0,x\in\partial\Omega\\
        \end{aligned}
    \right.
\end{equation}
The eigen-functions $\{\omega_{j}\}$ forms a complete orthonormal basis on $L^{2}(\Omega)$. As $\innerprod{u}{\omega_{1}}=0$, by Fourier extension, we can see:
\begin{equation}
    \label{eq:FourierExtend}
    u=\sum_{j=2}^{\infty}d_{j}\omega_{j}.
\end{equation}
Then:
\begin{equation}
    \label{eq:secondEigen}
    \frac{\innerprod{-\Delta u}{u}}{\|u\|^2}=\frac{\innerprod{\sum_{j=2}^{\infty}\lambda_{j}d_{j}\omega_{j}}{\sum_{j=2}^{\infty}d_{j}\omega_{j}}}{\sum_{j=2}^{\infty}d_{j}^{2}}=\frac{\sum_{j=2}^{\infty}\lambda_{j}d_{j}^{2}}{\sum_{j=2}^{\infty}d_{j}^{2}}\ge\lambda_{2}.
\end{equation}
If we choose $u=\omega_{2}$, \eqref{eq:secondEigen} equals to $\lambda_{2}$. So:
\begin{equation}
    \lambda_{2}=\inf_{u\in H_{0}^{1}(\Omega),\innerprod{u}{\omega_{1}}=0}\frac{\innerprod{-\Delta u}{u}}{\|u\|^{2}}.
\end{equation}
\end{document}