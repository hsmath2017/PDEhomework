\documentclass[a4paper]{ctexart}
\usepackage{geometry}
% useful packages.
\usepackage{amsfonts}
\usepackage{amsmath}
\usepackage{amssymb}
\usepackage{amsthm}
\usepackage{mathrsfs}
\usepackage{enumerate}
\usepackage{graphicx}
\usepackage{multicol}
\usepackage{fancyhdr}
\usepackage{layout}
\newtheorem{Definition}{\hspace{2em}定义}
\newtheorem{Example}{\hspace{2em}例}
\newtheorem{Thm}{\hspace{2em}定理}
\newtheorem{Lem}{\hspace{2em}引理}
\newtheorem{cor}{\hspace{2em}推论}
% some common command
\newcommand{\dif}{\mathrm{d}}
\newcommand{\avg}[1]{\left\langle #1 \right\rangle}
\newcommand{\difFrac}[2]{\frac{\dif #1}{\dif #2}}
\newcommand{\pdfFrac}[2]{\frac{\partial #1}{\partial #2}}
\newcommand{\OFL}{\mathrm{OFL}}
\newcommand{\UFL}{\mathrm{UFL}}
\newcommand{\fl}{\mathrm{fl}}
\newcommand{\op}{\odot}
\newcommand{\cp}{\cdot}
\newcommand{\Eabs}{E_{\mathrm{abs}}}
\newcommand{\Erel}{E_{\mathrm{rel}}}
\newcommand{\DR}{\mathcal{D}_{\widetilde{LN}}^{n}}
\newcommand{\add}[2]{\sum_{#1=1}^{#2}}
\newcommand{\innerprod}[2]{\left<#1,#2\right>}
\newcommand\tbbint{{-\mkern -16mu\int}}
\newcommand\tbint{{\mathchar '26\mkern -14mu\int}}
\newcommand\dbbint{{-\mkern -19mu\int}}
\newcommand\dbint{{\mathchar '26\mkern -18mu\int}}
\newcommand\bint{
{\mathchoice{\dbint}{\tbint}{\tbint}{\tbint}}
}
\newcommand\bbint{
{\mathchoice{\dbbint}{\tbbint}{\tbbint}{\tbbint}}
}
\title{Homework\# 8}
\author{Shuang Hu(26)}
\begin{document}
\maketitle
\section*{P41 Problem1}
(2)

\begin{Lem}
    $F(e^{-|x|})=\frac{2}{1+\xi^{2}}$.
\end{Lem}
\begin{proof}
    \begin{equation}
        \begin{aligned}
            F(e^{-|x|})&=\int_{\mathbb{R}}e^{-|x|}e^{i\xi x}\dif x\\
            &=\int_{0}^{+\infty}e^{x(i\xi-1)}\dif x+\int_{-\infty}^{0}e^{x(i\xi+1)}\dif x\\
            &=\frac{2}{1+\xi^{2}}.
        \end{aligned}
    \end{equation}
\end{proof}
So: 
\begin{equation}
    F(e^{-a|x|})=\frac{2a}{a^{2}+\xi^{2}}.
\end{equation}

Then: by inversion formula, we can see:
\begin{equation}
    e^{-a|x|}=\frac{1}{2\pi}\int_{\mathbb{R}}\frac{2a}{a^{2}+\xi^{2}}e^{i\xi x}\dif\xi.
\end{equation}

set $x\leftarrow -x$, we can see:
\begin{equation}
    F(\frac{1}{a^{2}+x^{2}})=\frac{\pi}{a}e^{-a|\xi|}.
\end{equation}

(4)
As:
\begin{equation}
    F(1)=2\pi\delta, F(x)=-DF(1).
\end{equation}

We can see:
\begin{equation}
    F(2x^{2}+x+1)=\delta-D\delta+2D^{2}\delta.
\end{equation}

(6)
\begin{equation}
    \log|x|=D(P.V.(\frac{1}{x})).
\end{equation}
So:
\begin{equation}
    F(\log|x|)=\xi F(P.V.(\frac{1}{x}))=-i\pi\xi\text{sgn}\xi
\end{equation}
\section*{P41 Problem3}
\begin{proof}
    $(1)\Leftrightarrow(2)$: according to the property of Fourier transformation, we can see:
    \begin{equation}
        \label{eq:DirFT}
        F(D^{\alpha}f)=\xi^{\alpha}\hat{f}(\xi).
    \end{equation}

    By Theorem 3.2(Parseval equality), we can see:
    \begin{equation}
        \label{eq:Parseval}
        \int_{\mathbb{R}^{n}}|D^{\alpha}f|^{2}\dif x=(2\pi)^{-n}\int_{\mathbb{R}^{n}}|F[D^{\alpha}f]|^{2}\dif x.
    \end{equation}

    According to \eqref{eq:DirFT} and \eqref{eq:Parseval}, we cna see $(1)\Leftrightarrow(2)$.

    For $(3)\Rightarrow(4)$, set 
    \begin{equation}
        h(\xi)=\left\{
            \begin{aligned}
                2,&|\xi|\le1\\
                2|\xi|^{2},&|\xi|>1.
            \end{aligned}
        \right.
    \end{equation}
    we can see $(1+|\xi|^{2})^{\frac{m}{2}}\le(h(\xi))^{\frac{m}{2}}$.

    $\forall P(\xi)$, $P(\xi)\hat{f}(\xi)\in L^{2}(\mathbb{R}^{n})$, so:
    \begin{equation}
        \begin{aligned}
            \|(1+\xi^{2})^{\frac{m}{2}}\hat{f}(\xi)\|_{L^{2}}&\le\|\hat{f}(\xi)\|_{L^{2}}+\||\xi|^{m}\hat{f}(\xi)\|_{L^{2}}\\
            &<+\infty.
        \end{aligned}
    \end{equation}
    
    It means $(3)\Rightarrow (4)$.

    Then we need to show that $(4)\Rightarrow (2)$. By mean-value inequality, for $|\alpha|\le m$, we can see:
    \begin{equation}
        |\xi^{\alpha}|\le(1+|\xi|^{2})^{\frac{m}{2}}.
    \end{equation}

    So 
    \begin{equation}
        \|\xi^{\alpha}\hat{f}(\xi)\|_{L^{2}}\le\|(1+|\xi|^{2})^{\frac{m}{2}}\hat{f}(\xi)\|_{L^{2}}.
    \end{equation}

    It means that $(4)\Rightarrow(2)$. Above all, $(1),(2),(3),(4)$ are all equivalent.
\end{proof}
\section*{P41 Problem4}
\begin{proof}
    As $S\in\mathscr{S}'(\mathbb{R}^{n})$, $T\in\mathscr{E}'(\mathbb{R}^{n})$, we can see $T*S\in\mathscr{S}'$. 

    Let $(\alpha_{j})$ be a sequence in $C_{c}^{\infty}(\mathbb{R}^{n})$ converging to $\delta$ in $\mathscr{E}'(\mathbb{R}^{n})$ as $j\rightarrow\infty$, we can see the sequence of functions 
    \begin{equation}
        \phi_{j}=\alpha_{j}*T\in C_{c}^{\infty}
    \end{equation}
    converges to $T$ in $\mathscr{\epsilon}'$. Then $\phi_{j}*S\rightarrow T*S$ in $\mathscr{S}'$. Hence, taking Fourier transforms:
    \begin{equation}
        F[T*S]=\lim_{j\rightarrow\infty}F[\phi_{j}*S]=\lim_{j\rightarrow\infty}F[\phi_{j}]F[S].
    \end{equation}

    On the other hand, since $\phi_{j}\rightarrow T$ in $\mathscr{E}'$, it also converges in $\mathscr{S}'$. Hence, by Fourier transform $F[\phi_{j}]\rightarrow F[T]$ in $\mathscr{S}'$.
    So it means that 
    \begin{equation}
        \lim_{j\rightarrow\infty}F[\phi_{j}]F[S]=F[T]F[S].
    \end{equation}

    Finally, we can see that
    \begin{equation}
        F[T*S]=F[T]F[S].
    \end{equation}
\end{proof}
\section*{P42 Problem7}
(3)
\begin{equation}
    \begin{aligned}
        F(xe^{-\pi y^2})&=F_{x}(x)F_{y}(e^{-\pi y^2})\\
        &=-2\pi\delta'(\xi)e^{-\frac{\eta^{2}}{4\pi}}\\.    
    \end{aligned}
\end{equation}

(4)
\begin{equation}
    \begin{aligned}
        F(\delta'(x)e^{-\frac{y^2}{2}})&=F_{x}(\delta'(x))F_{y}(e^{-\frac{y^2}{2}})\\
        &=\sqrt{2\pi}\xi e^{-\frac{|\eta|^{2}}{2}}.\\
    \end{aligned}
\end{equation}
\section*{P42 Problem8}
题都没看懂...
\section*{P42 Problem9}
\begin{proof}
    By definition, $Au(x)=\int e^{i\innerprod{x}{\xi}}a(x,\xi)\hat{u}(\xi)\dif\xi$. Fourier transformation is a linear map from $\mathscr{S}(\mathbb{R}^{n})$ to $\mathscr{S}(\mathbb{R}^{n})$, so $\hat{u}(\xi)\in\mathscr{S}(\mathbb{R}^{n})$. Now we consider the expression 
    $\partial^{p}Au(x)$.

    We can see:
    \begin{equation}
        \int |\partial^{p}(e^{i\innerprod{x}{\xi}}a(x,\xi)\hat{u}(\xi))|\dif\xi\le
        \int |e^{i\innerprod{x}{\xi}}|\xi|^{p}\hat{u}(\xi)|\dif\xi<\infty.
    \end{equation}

    So, by dominent convergent theorem(DCT), we can swap the order of integral and derivative. Now, consider 
    \begin{equation}
        Bu(x)=\int e^{i\innerprod{x}{\xi}}|\xi|^{p}\hat{u}(\xi)\dif\xi.
    \end{equation}
    As $\hat{u}\in\mathscr{S}$, this integral must be convergent, and $\exists$ constant $C$ such that $Bu(x)\le C^{p}u(x)$. $u\in\mathscr{S}\Rightarrow Au(x)\in\mathscr{S}$.
\end{proof}
\end{document}