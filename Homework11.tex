\documentclass[a4paper]{ctexart}
\usepackage{geometry}
% useful packages.
\usepackage{amsfonts}
\usepackage{amsmath}
\usepackage{amssymb}
\usepackage{amsthm}
\usepackage{mathrsfs}
\usepackage{enumerate}
\usepackage{graphicx}
\usepackage{multicol}
\usepackage{fancyhdr}
\usepackage{layout}
\newtheorem{Definition}{\hspace{2em}定义}
\newtheorem{Example}{\hspace{2em}例}
\newtheorem{Thm}{\hspace{2em}定理}
\newtheorem{Lem}{\hspace{2em}引理}
\newtheorem{cor}{\hspace{2em}推论}
% some common command
\newcommand{\dif}{\mathrm{d}}
\newcommand{\avg}[1]{\left\langle #1 \right\rangle}
\newcommand{\difFrac}[2]{\frac{\dif #1}{\dif #2}}
\newcommand{\pdfFrac}[2]{\frac{\partial #1}{\partial #2}}
\newcommand{\OFL}{\mathrm{OFL}}
\newcommand{\UFL}{\mathrm{UFL}}
\newcommand{\fl}{\mathrm{fl}}
\newcommand{\op}{\odot}
\newcommand{\cp}{\cdot}
\newcommand{\Eabs}{E_{\mathrm{abs}}}
\newcommand{\Erel}{E_{\mathrm{rel}}}
\newcommand{\DR}{\mathcal{D}_{\widetilde{LN}}^{n}}
\newcommand{\add}[2]{\sum_{#1=1}^{#2}}
\newcommand{\innerprod}[2]{\left<#1,#2\right>}
\newcommand\tbbint{{-\mkern -16mu\int}}
\newcommand\tbint{{\mathchar '26\mkern -14mu\int}}
\newcommand\dbbint{{-\mkern -19mu\int}}
\newcommand\dbint{{\mathchar '26\mkern -18mu\int}}
\newcommand\bint{
{\mathchoice{\dbint}{\tbint}{\tbint}{\tbint}}
}
\newcommand\bbint{
{\mathchoice{\dbbint}{\tbbint}{\tbbint}{\tbbint}}
}
\title{Homework\# 11}
\author{Shuang Hu(26)}
\begin{document}
\maketitle
\section*{P76 Problem2}
\begin{proof}
    By definition:
    \begin{equation}
        p_{m}(x,\xi)=(i\xi_{t})^{2}-a^2((i\xi_{x})^2+(i\xi_{y})^{2})
        =-\xi_{t}^2+a^2(\xi_{x}^2+\xi_{y}^2).
    \end{equation}
    The cone: $t^2=a^2(x^2+y^2)$, we choose a direction outside this cone, the direction vector $l=(r,\cos\theta,\sin\theta)$, while $r>a$. Choose vector $\xi$ is not parallel to $l$, then check the function $p_{m}(x,\lambda l+\xi)$, we can see:
    \begin{equation}
        \label{eq:character}
        g(\lambda)=p_{m}(x,\lambda l+\xi)=(a^2-r^2)\lambda^2+(2a\xi_{y}\sin\theta+2a\xi_{x}\cos\theta-2\xi_{t}r)\lambda+(a^2\xi_{x}^2+a^2\xi_{y}^2-\xi_{t}^{2}).
    \end{equation}
    As the direction $\xi$ lies outside the given cone, it's clear that:
    \begin{equation}
        a^2\xi_{x}^2+a^2\xi_{y}^2-\xi_{t}^{2}>0.
    \end{equation}
    And $r>a$ means $a^2-r^2<0$. Then we derive the discriminant of \eqref{eq:character}:
    \begin{equation}
        \Delta=(2a\xi_{y}\sin\theta+2a\xi_{x}\cos\theta-2\xi_{t}r)^2-4(a^2-r^2)(a^2\xi_{x}^2+a^2\xi_{y}^2-\xi_{t}^{2})>0.
    \end{equation}
    It means $p_{m}(x,\xi)$ always has two different roots. Then this equation is strictly hyperbolic.
\end{proof}
\section*{P77 Problem4}
The characteristic surface $\phi(t,x,y)=0$ satisfies two conditions:
\begin{equation}
    \label{eq:char-surface}
    \left\{
        \begin{aligned}
           \phi_{t}^2-a^2(\phi_{x}^{2}+\phi_{y}^{2})&=0\\
           \phi(0,3\lambda,\lambda)&=0\\ 
        \end{aligned}
    \right.
\end{equation}
Solve this ODEs, we can see the characteristic surface is 
\begin{equation}
    x-3y+\sqrt{10}at=0,
\end{equation}
or
\begin{equation}
    x-3y-\sqrt{10}at=0.
\end{equation}
\section*{P106 Problem1}
Assume $u$ is the foundamental solution of the operator 
\begin{equation}
    \label{eq:DifOp}
    P(\mathrm{d})=\difFrac{^2}{x^2}+c\difFrac{}{x}.
\end{equation}
Then we can see $P(\mathrm{d})u=\delta$. Then:
\begin{equation}
    \label{eq:Foundamental}
    \begin{aligned}
        &\mathrm{d}(e^{cx}\difFrac{u}{x})=e^{cx}\delta=\delta\\
        \Rightarrow&e^{cx}\difFrac{u}{x}=H(x)\\
        \Rightarrow&\difFrac{u}{x}=e^{-cx}H(x)\\
        \Rightarrow&u(x)=\left\{
            \begin{aligned}
                0,&x\le 0\\
                \frac{1-e^{-cx}}{c},&x>0\\
            \end{aligned}
        \right.
    \end{aligned}
\end{equation}
So $u(x)$ is the foundamental solution of the operator $P(\mathrm{d})$.
\section*{P106 Problem3}
Assume the foundamental solution is $T(x,y)$, do Fourier transformation on $T(x,y)$ related to $y$, mark $\hat{T}(x)=F_{y}[T]$, we can see:
\begin{equation}
    \label{eq:FTforT}
    \begin{aligned}
        &\difFrac{\hat{T}}{x}-\eta\hat{T}+c\hat{T}=\delta(x)\\
        \Rightarrow&\hat{T}=e^{(\eta-c)x}H(x)+C(\eta)e^{(\eta-c)x}.\\
    \end{aligned}
\end{equation}
We should choose the form of $C(\eta)$ to make $\hat{T}$ an $\mathscr{S}'$-contribution, just set:
\begin{equation}
    \label{eq:chooseeta}
    C(\eta)=\left\{
        \begin{aligned}
            -1,&\eta>c\\
            0,&\eta<c\\
        \end{aligned}
    \right.
\end{equation}
We can see:
\begin{equation}
    \label{eq:hatT}
    \hat{T}=\left\{
        \begin{aligned}
            -H(-x)e^{(\eta-c)x},&\eta>c\\
            H(x)e^{(\eta-c)x},&\eta<c\\
        \end{aligned}
    \right.
\end{equation}
Do inverse FT on \eqref{eq:hatT}, we can see when $x\neq 0$,
\begin{equation}
    \begin{aligned}
        T(x,y)&=\frac{H(x)}{2\pi}\int_{-\infty}^{c}e^{(\eta-c)x}e^{iy\eta}\dif\eta
        -\frac{H(-x)}{2\pi}\int_{c}^{+\infty}e^{(\eta-c)x}e^{iy\eta}\dif\eta.\\
        &=\frac{e^{icy}}{2\pi}\frac{1}{x+iy}.
    \end{aligned}
\end{equation}
\section*{P106 Problem5}
For the foundamental solution $E(x,t)$, we can see:
\begin{equation}
    \label{eq:FoundamentalSolutionForI-D}
    (I-\Delta)E=\delta.
\end{equation}
Do Fourier transformation on \eqref{eq:FoundamentalSolutionForI-D}, we can see:
\begin{equation}
    \hat{E}(\xi)=\frac{1}{1+|\xi|^2}.
\end{equation}
It means that:
\begin{equation}
    E(x)=\int_{\mathbb{R}^{n}}\frac{e^{i\xi\cdot x}}{1+|\xi|^2}\dif\xi.
\end{equation}
$E(x)$ is $C^{\infty}$ on $\mathbb{R}^{n}\setminus\{0\}$, so $I-\Delta$ is semi-elliptic.
\section*{P106 Problem6}
Do Fourier transformation on $u(x,t)$ related to $x$, mark the result as $\hat{u}$, we can see:
\begin{equation}
    \left\{
        \begin{aligned}
            &\difFrac{^2\hat{u}}{t^2}+\xi^2\hat{u}=0,\\
            &\hat{u}(\xi,0)=0,\\
            &\difFrac{\hat{u}}{t}(\xi,0)=1.\\
        \end{aligned}
    \right.
\end{equation}
It means that 
\begin{equation}
    \label{eq:FTofSol}
    \hat{u}(\xi,t)=\frac{\sin |\xi| t}{|\xi|}.
\end{equation}
Then, we should get the inverse FT of \eqref{eq:FTofSol}. It's clear that:
\begin{equation}
    \label{eq:FoundamentalOfWave1D}
    \begin{aligned}
        u(x,t)&=\frac{1}{2\pi}\int_{\mathbb{R}}\frac{\sin(|\xi| t)}{|\xi|}e^{i\xi x}\dif\xi\\
        &=\frac{1}{\pi}\int_{0}^{\infty}\frac{\sin(\xi t)\cos(\xi x)}{\xi}\dif\xi\\
        &=\frac{1}{4}(\text{sgn}(t+x)+\text{sgn}(t-x))\\
    \end{aligned}
\end{equation}
To derive the D'Alembert's formula, consider the BVP of wave function:
\begin{equation}
    \label{eq:Wave}
    \left\{
        \begin{aligned}
            \pdfFrac{^2u}{t^2}-\pdfFrac{^2u}{x^2}&=\delta(t,x)\\
            u|_{t=0}&=0,\\
            \pdfFrac{u}{t}|_{t=0}&=\varphi(x).\\
        \end{aligned}
    \right.
\end{equation}
Mark the foundamental solution \eqref{eq:FoundamentalOfWave1D} as $E(x,t)$, the solution of wave equation \eqref{eq:Wave} is:
\begin{equation}
    u(x,t)=E(x,t)*\varphi(x)=\frac{1}{2t}\int_{x-t}^{x+t}\varphi(\xi)\dif\xi.
\end{equation}
Then by DuHamel's Principle, D'Alembert formula is true.
\end{document}