\documentclass[a4paper]{ctexart}
% useful packages.
\usepackage{amsfonts}
\usepackage{amsmath}
\usepackage{amssymb}
\usepackage{amsthm}
\usepackage{enumerate}
\usepackage{graphicx}
\usepackage{multicol}
\usepackage{fancyhdr}
\usepackage{layout}
\newtheorem{Definition}{\hspace{2em}定义}
\newtheorem{Example}{\hspace{2em}例}
\newtheorem{Thm}{\hspace{2em}定理}
\newtheorem{Lem}{\hspace{2em}引理}
\newtheorem{cor}{\hspace{2em}推论}
% some common command
\newcommand{\dif}{\mathrm{d}}
\newcommand{\avg}[1]{\left\langle #1 \right\rangle}
\newcommand{\difFrac}[2]{\frac{\dif #1}{\dif #2}}
\newcommand{\pdfFrac}[2]{\frac{\partial #1}{\partial #2}}
\newcommand{\OFL}{\mathrm{OFL}}
\newcommand{\UFL}{\mathrm{UFL}}
\newcommand{\fl}{\mathrm{fl}}
\newcommand{\op}{\odot}
\newcommand{\cp}{\cdot}
\newcommand{\Eabs}{E_{\mathrm{abs}}}
\newcommand{\Erel}{E_{\mathrm{rel}}}
\newcommand{\DR}{\mathcal{D}_{\widetilde{LN}}^{n}}
\newcommand\tbbint{{-\mkern -16mu\int}}
\newcommand\tbint{{\mathchar '26\mkern -14mu\int}}
\newcommand\dbbint{{-\mkern -19mu\int}}
\newcommand\dbint{{\mathchar '26\mkern -18mu\int}}
\newcommand\bint{
{\mathchoice{\dbint}{\tbint}{\tbint}{\tbint}}
}
\newcommand\bbint{
{\mathchoice{\dbbint}{\tbbint}{\tbbint}{\tbbint}}
}
\title{Homework\# 3}
\author{Shuang Hu(26)}
\begin{document}
\maketitle
\section*{P85 Problem1}
\begin{proof}
    Set $w(t):=u(\phi(x,t),t)$ such that
    \begin{equation}
        \difFrac{w}{t}=\pdfFrac{u}{t}+b\cdot Du.
    \end{equation}
    it means that $\pdfFrac{\phi}{t}=b$, which means $\phi(x,t)=x_{0}+bt$. After this modify, 
    we can see:
    \begin{equation}
        \left\{
            \begin{aligned}
                \difFrac{w}{t}+cw&=0\\
                w_{x_{0}}(0)&=u(x_{0},0)\\
            \end{aligned}
        \right.
    \end{equation}
    this ODE suggests:
    \begin{equation}
        w(t)=e^{-ct}u(x_{0},0).
    \end{equation}
    So:
    \begin{equation}
        u(x,t)=e^{-ct}u(x-bt,0)=e^{-ct}g(x-bt).
    \end{equation}
\end{proof}
\section*{P88 Problem18}
\begin{proof}
    \begin{equation}
        \left.
            \begin{aligned}
                v_{tt}&=u_{ttt}\\
                \Delta v&=\Delta u_{t}\\
            \end{aligned}
        \right\}
        \Rightarrow\pdfFrac{(u_{tt}-\Delta u)}{t}=0\Rightarrow v_{tt}-\Delta v=0.
    \end{equation}
    on $\mathbb{R}^{n}\times\{t=0\}$, we can see $v=u_{t}=h$, and $v_{t}(x,0)=\lim_{t\rightarrow 0}\Delta u(x,t)$.

    For $u\equiv 0$ when $t=0$, we can see $\Delta u(x,0)=0$. As $\Delta u$ is continuous, we can see $v_{t}(x,0)=0$.
\end{proof}
\section*{P88 Problem19}
(a)
\begin{equation}
    \begin{aligned}
        u_{xy}=0&\Rightarrow\pdfFrac{u_{x}}{y}=0\Rightarrow u_{x}=f(x)+C_{1}.\\
        u_{xy}=0&\Rightarrow\pdfFrac{u_{y}}{x}=0\Rightarrow u_{y}=g(y)+C_{2}.\\
    \end{aligned}
\end{equation}
It means that $u(x,y)=F(x)+G(y)+C$.

(b)
\begin{equation}
    \begin{aligned}
        \pdfFrac{u_{\xi}}{\eta}&=\pdfFrac{u_{\xi}}{t}\pdfFrac{t}{\eta}+\pdfFrac{u_{\xi}}{x}\pdfFrac{x}{\eta}\\
        &=u_{\xi x}-u_{\xi t}\\
        &=\pdfFrac{u_{x}}{\xi}-\pdfFrac{u_{t}}{\xi}\\
        &=(\pdfFrac{u_{x}}{x}\pdfFrac{x}{\xi}+\pdfFrac{u_{x}}{t}\pdfFrac{t}{\xi})-(\pdfFrac{u_{t}}{x}\pdfFrac{x}{\xi}+\pdfFrac{u_{t}}{t}\pdfFrac{t}{\xi})\\
        &=u_{xx}+u_{xt}-u_{xt}-u_{tt}=u_{xx}-u_{tt}.
    \end{aligned}
\end{equation}

(c)
\begin{equation}
    \begin{aligned}
    u_{tt}-u_{xx}=0\Rightarrow u_{\xi\eta}=0&\Rightarrow u(\xi,\eta)=f(\xi)+g(\eta)\\
    &\Rightarrow u(x,t)=F(x-t)+G(x+t).\\
    \end{aligned}
\end{equation}
\begin{equation}
    \left\{
        \begin{aligned}
            u(x,0)&=F(x)+G(x)=g(x)\\
            u_{t}(x,0)&=-F'(x)+G'(x)=h(x)\\
        \end{aligned}
    \right.
    \Rightarrow
    \left\{
        \begin{aligned}
            G'(x)&=\frac{h(x)+g'(x)}{2}\\
            F'(x)&=\frac{g'(x)-h(x)}{2}\\
        \end{aligned}
    \right.
\end{equation}
It means that 
\begin{equation}
    u(x,t)=\frac{1}{2}(g(x+t)+g(x-t))+\frac{1}{2}\int_{x-t}^{x+t}h(y)\dif y.
\end{equation}

(d)

Right-moving: $g(\xi)\equiv 0$ when $\xi<x$.

Left-moving: $g(\xi)\equiv 0$ when $\xi>x$. 
\section*{P89 Problem20}
\begin{proof}
    If $u$ satisfies the wave equation, it means:
    \begin{equation}
        \label{eq:wave20}
        \sum_{i=1}^{n}u_{x_{i}x_{i}}=\frac{1}{c^{2}}u_{tt}.
    \end{equation}
    Set $r=|x|$, \ref{eq:wave20} means that 
    \begin{equation}
        \label{eq:funcv20}
        v_{rr}+\frac{n-1}{r}v=\frac{1}{c^{2}}v_{tt}.
    \end{equation}
    Suppose $u=\alpha(r)\phi(t-\beta(r))$, according to \ref{eq:funcv20}, we can see:
    \begin{equation}
        \label{eq:ap20}
        \alpha''\phi-2\alpha'\beta'\phi'-\alpha\beta''\phi'+\alpha(\beta')^{2}\phi''+\frac{n-1}{r}(\alpha'\phi-\alpha\phi'\beta')=\frac{\alpha}{c^{2}}\phi''
    \end{equation}
    According to \ref{eq:ap20} and $\beta(0)=0$, we can see:
    \begin{equation}
        \label{eq:2phi20}
        \alpha(\beta')^{2}=\frac{\alpha}{c}\Rightarrow \beta=\frac{r}{c},\beta'=\frac{1}{c},\beta''=0.
    \end{equation}
    and:
    \begin{equation}
        \label{eq:phi20}
        \begin{aligned}
            \alpha''+\frac{n-1}{r}\alpha'&=0\\
            2\alpha'+\frac{n-1}{r}\alpha&=0\\
        \end{aligned}
    \end{equation}
    From the first equation in \ref{eq:phi20}, we can see $\alpha'=\frac{C}{r^{n-1}}$. It means that if $n\neq 2$, 
    $\alpha=C_{1}r^{2-n}+C_{2}$, if $n=2$, $\alpha=C_{1}\log(r)+C_{2}$.
    If it satisfies the second equation in \ref{eq:phi20}, it means:
    \begin{equation}
        \label{eq:alpha20}
        2(2-n)+(n-1)=0\Rightarrow n=3.
    \end{equation}
    In special case ,when $n=1$, the equation \ref{eq:phi20} is also correct. So $n=1$ or $n=3$.

    When $n=1$, $\alpha=C$, $\beta=\frac{r}{c}$. When $n=3$, $\alpha=\frac{k}{r}$, $\beta=\frac{r}{c}$.
\end{proof}
\section*{P89 Problem21}
(a)
\begin{equation}
    \label{eq:Maxwell}
    \begin{aligned}
        E_{tt}&=\text{curl}(B_{t})\\
        &=\text{curl}(-\text{curl}E)\\
        &=\text{curl}(\pdfFrac{E_{2}}{x_{3}}-\pdfFrac{E_{3}}{x_{2}},\pdfFrac{E_{3}}{x_{1}}-\pdfFrac{E_{1}}{x_{3}},\pdfFrac{E_{1}}{x_{2}}-\pdfFrac{E_{2}}{x_{1}}).\\
        &=\Delta E.
    \end{aligned}
\end{equation}
The final equation is derived by the condition $\nabla\cdot E=0$. In the same way, we can see:$B_{tt}=\Delta B$.

(b)

For $w=\nabla\cdot u$, get divergence on the given equation, we can see:
\begin{equation}
    \begin{aligned}
        &\nabla\cdot u_{tt}-\mu\nabla\cdot\Delta u-(\lambda+\mu)\nabla\cdot(\nabla(\nabla\cdot u))=0\\
        \Rightarrow&w_{tt}-(\lambda+\mu)w-\mu\nabla(\Delta u)=0\\
        \Rightarrow& w_{tt}-(\lambda+2\mu)\Delta w=0.
    \end{aligned}
\end{equation}
For $w=\nabla\times u$, in the same way, we can see $w_{tt}-\mu\Delta w=0$.
\end{document}