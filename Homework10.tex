\documentclass[a4paper]{ctexart}
\usepackage{geometry}
% useful packages.
\usepackage{amsfonts}
\usepackage{amsmath}
\usepackage{amssymb}
\usepackage{amsthm}
\usepackage{mathrsfs}
\usepackage{enumerate}
\usepackage{graphicx}
\usepackage{multicol}
\usepackage{fancyhdr}
\usepackage{layout}
\newtheorem{Definition}{\hspace{2em}定义}
\newtheorem{Example}{\hspace{2em}例}
\newtheorem{Thm}{\hspace{2em}定理}
\newtheorem{Lem}{\hspace{2em}引理}
\newtheorem{cor}{\hspace{2em}推论}
% some common command
\newcommand{\dif}{\mathrm{d}}
\newcommand{\avg}[1]{\left\langle #1 \right\rangle}
\newcommand{\difFrac}[2]{\frac{\dif #1}{\dif #2}}
\newcommand{\pdfFrac}[2]{\frac{\partial #1}{\partial #2}}
\newcommand{\OFL}{\mathrm{OFL}}
\newcommand{\UFL}{\mathrm{UFL}}
\newcommand{\fl}{\mathrm{fl}}
\newcommand{\op}{\odot}
\newcommand{\cp}{\cdot}
\newcommand{\Eabs}{E_{\mathrm{abs}}}
\newcommand{\Erel}{E_{\mathrm{rel}}}
\newcommand{\DR}{\mathcal{D}_{\widetilde{LN}}^{n}}
\newcommand{\add}[2]{\sum_{#1=1}^{#2}}
\newcommand{\innerprod}[2]{\left<#1,#2\right>}
\newcommand\tbbint{{-\mkern -16mu\int}}
\newcommand\tbint{{\mathchar '26\mkern -14mu\int}}
\newcommand\dbbint{{-\mkern -19mu\int}}
\newcommand\dbint{{\mathchar '26\mkern -18mu\int}}
\newcommand\bint{
{\mathchoice{\dbint}{\tbint}{\tbint}{\tbint}}
}
\newcommand\bbint{
{\mathchoice{\dbbint}{\tbbint}{\tbbint}{\tbbint}}
}
\title{Homework\# 10}
\author{Shuang Hu(26)}
\begin{document}
\maketitle
\section*{Evans P308 Problem7}
By the given condition and \textbf{Gauss-Green Formula}, we can see:
\begin{equation}
    \label{eq:GaussGreenStep}
    \begin{aligned}
        \int_{\partial U}|u|^{p}\dif S&\le\int_{\partial U}|u|^{p}\alpha\cdot\mu\dif S\\
        &=\int_{U}|u|^{p}(\nabla\cdot\alpha)\dif x+\int_{U}\alpha\cdot(\nabla|u|^{p})\dif x\\
        &\le C\int_{U}(|u|^{p}+|\nabla|u|^{p}|)\dif x.
    \end{aligned}
\end{equation}
Since 
\begin{equation}
    \label{eq:nablaup}
    \nabla|u|^{p}=p|u|^{p-1}(\text{sgn}u)\nabla u,
\end{equation}
we have for $p=1$, 
\begin{equation}
    \int_{\partial U}|u|\dif S\le C\int_{U}(|u|+|\nabla u|)\dif x.
\end{equation}
If $p>1$, by \eqref{eq:nablaup}, we can see:
\begin{equation}
    \label{eq:approximateInt}
    \int_{U}|\nabla|u|^{p}|\dif x\le C\int_{U}p|u|^{p-1}|\nabla u|\dif x.
\end{equation}
Then by \textbf{Young's Inequality}, we can see:
\begin{equation}
    \label{eq:Young}
    |\nabla u||u|^{p-1}\le\frac{|\nabla u|^{p}}{p}+\frac{|u|^{q(p-1)}}{q}
    \le \frac{|\nabla u|^{p}}{p}+\frac{|u|^{p}}{q}.
\end{equation}
Combine \eqref{eq:GaussGreenStep}, \eqref{eq:nablaup} and \eqref{eq:Young}, we can see:
\begin{equation}
    \int_{\partial U}|u|^{p}\dif S\le \text{Const}\int_{U}(|\nabla U|^{p}+|u|^{p})\dif x.
\end{equation}
\section*{P309 Problem8}
\begin{proof}
    If $T$ is a bounded and linear operator, by the definition, we have:
    \begin{equation}
        \label{eq:BoundedOp}
        \|Tu\|\le C\|u\|\forall u\in L^{p}(\Omega).
    \end{equation}
    It means that:
    \begin{equation}
        \label{eq:OperatorIneq}
        \int_{\partial \Omega}|u|^{p}\dif x\le C\int_{\Omega}|u|^{p}\dif x,\forall u\in L^{p}(\Omega).
    \end{equation}
    In fact, $\forall \epsilon>0$, there exists $u\in C^{\infty}(\Omega)$ such that $u|_{\partial\Omega}\equiv 1$, $u(x)=0$ when $\text{dist}(x,\partial\Omega)>\epsilon$. When $\epsilon\rightarrow 0$, the left side of \eqref{eq:OperatorIneq} equivalent the length of $\partial\Omega$, and the right side of \eqref{eq:OperatorIneq} converges to 0, contradict!

    So $T$ isn't bounded in general.
\end{proof}
\section*{P309 Problem10}
(a)
\begin{proof}
    The first step is an integration by parts:
    \begin{equation}
        \label{eq:intbypart}
        \begin{aligned}
        \int_{U}|Du|^{p}\dif x&=\int_{U}\nabla u\cdot\nabla u|Du|^{p-2}\dif x\\
        &=-\int_{U}u\nabla\cdot(\nabla u|Du|^{p-2})\dif x\\
        &=-\int_{U}u(\Delta u|Du|^{p-2}+(p-2)(\nabla u^{T}D^{2}u\nabla u)|Du|^{p-4})\\
        &\le C\int_{U}u|Du|^{p-2}|D^{2}u|\dif x\\
        \end{aligned}
    \end{equation}    
    Then, as $p\neq 2$, by Holder Inequality, we can see:
    \begin{equation}
        \label{eq:holder}
        \int_{U}u|Du|^{p-2}|D^{2}u|\dif x\le(\int_{U}|u|^{\frac{p}{2}}|D^{2}u|^{\frac{p}{2}}\dif x)^{\frac{2}{p}}(\int_{U}|Du|^{p}\dif x)^{\frac{p-2}{p}}.
    \end{equation}
    By \eqref{eq:intbypart}, \eqref{eq:holder} and Cauchy inequality, we can see:
    \begin{equation}
        \int_{U}|Du|^{p}\dif x\le C(\int_{U}|u|^{p}\dif x)^{\frac{1}{2}}(\int_{U}|D^{2}u|^{p}\dif x)^{\frac{1}{2}}.
    \end{equation}
\end{proof}
(b)
\begin{proof}
    by (a), we can see:
    \begin{equation}
        \label{eq:IntByPartsB}
        \int_{U}|Du|^{2p}\dif x\le C\int_{U}u|Du|^{2p-2}|D^{2}u|\dif x.
    \end{equation}
    Then by Holder inequality:
    \begin{equation}
        \label{eq:CS}
        \begin{aligned}
            \eqref{eq:IntByPartsB}&\le C\|u\|_{L^\infty}\int_{U}|Du|^{2p-2}|D^{2}u|\dif x\\
            &\le C\|u\|_{L^{\infty}}(\int_{U}|Du|^{2p}\dif x)^{\frac{p-1}{p}}(\int_{U}|D^{2}u|^{p}\dif x)^{\frac{1}{p}}\\
        \end{aligned}
    \end{equation}
    By \eqref{eq:IntByPartsB} and \eqref{eq:CS}, 
    \begin{equation}
        \|Du\|_{L^{2p}}\le C\|u\|_{L^{\infty}}^{\frac{1}{2}}\|D^{2}u\|_{L^{p}}^{\frac{1}{2}}.
    \end{equation}
\end{proof}
\section*{P309 Problem14}
\begin{proof}
Mark $w_{n}$ as the area of $n$-dimension ball, we can see:
\begin{equation}
    \label{eq:IntOfFunc}
    \begin{aligned}
        \int_{U}|u|^{n}\dif x&=\omega_{n}\int_{0}^{1}|\log\log(1+\frac{1}{r})|^{n}r^{n-1}\dif r\\
        &=\omega_{n}\int_{1}^{\infty}\frac{1}{t^{n}}\frac{|\log\log(1+t)|^{n}}{t}\dif t\\
    \end{aligned}
\end{equation}
Since $\frac{|\log\log(1+t)|^{n}}{t}\rightarrow 0$ when $t\rightarrow \infty$, $\exists T$ such that $\frac{|\log\log(1+t)|^{n}}{t}<1$ is true $\forall t>T$. It means that:
\begin{equation}
    \label{eq:approxInt}
    \eqref{eq:IntOfFunc}\le \omega_{n}\int_{1}^{T}\frac{1}{t^{n}}\frac{|\log\log(1+t)|^{n}}{t}\dif t+\int_{1}^{\infty}\frac{1}{t^{n}}\dif t<+\infty.
\end{equation}
Then it's time to consider the case of derivatives. In fact:
\begin{equation}
    \label{eq:Derivative}
    \pdfFrac{u}{x_{i}}=\frac{1}{\log(1+\frac{1}{|x|})}.\frac{1}{1+\frac{1}{|x|}}.\frac{-x_{i}}{|x|^{3}}.
\end{equation}
By \eqref{eq:Derivative}:
\begin{equation}
    \begin{aligned}
        \int_{U}|u_{x_{i}}|^{n}\dif x&\le\omega_{n}\int_{0}^{1}(\frac{1}{\log(1+\frac{1}{r})}\frac{1}{1+\frac{1}{r}}\frac{1}{r^{2}})^{n}r^{n-1}\dif r\\
        &=\omega_{n}\int_{1}^{+\infty}\frac{1}{|\log(1+t)|^{n}}\frac{1}{(1+t)^{n}}t^{n-1}\dif t\\
        &\le\frac{\omega_{n}}{n}\int_{\log 2}^{+\infty}\frac{1}{s^{n}}\frac{1}{e^{sn}}e^{sn}\dif s\\
        &<\infty.
    \end{aligned}
\end{equation}
It means $u\in W^{1,n}(U)$.
\end{proof}
\section*{P311 Problem21}
\end{document}