\documentclass[a4paper]{ctexart}
% useful packages.
\usepackage{amsfonts}
\usepackage{amsmath}
\usepackage{amssymb}
\usepackage{amsthm}
\usepackage{enumerate}
\usepackage{graphicx}
\usepackage{multicol}
\usepackage{fancyhdr}
\usepackage{layout}
\newtheorem{Definition}{\hspace{2em}定义}
\newtheorem{Example}{\hspace{2em}例}
\newtheorem{Thm}{\hspace{2em}定理}
\newtheorem{Lem}{\hspace{2em}引理}
\newtheorem{cor}{\hspace{2em}推论}
% some common command
\newcommand{\dif}{\mathrm{d}}
\newcommand{\avg}[1]{\left\langle #1 \right\rangle}
\newcommand{\difFrac}[2]{\frac{\dif #1}{\dif #2}}
\newcommand{\pdfFrac}[2]{\frac{\partial #1}{\partial #2}}
\newcommand{\OFL}{\mathrm{OFL}}
\newcommand{\UFL}{\mathrm{UFL}}
\newcommand{\fl}{\mathrm{fl}}
\newcommand{\op}{\odot}
\newcommand{\cp}{\cdot}
\newcommand{\Eabs}{E_{\mathrm{abs}}}
\newcommand{\Erel}{E_{\mathrm{rel}}}
\newcommand{\DR}{\mathcal{D}_{\widetilde{LN}}^{n}}
\newcommand{\add}[2]{\sum_{#1=1}^{#2}}
\newcommand\tbbint{{-\mkern -16mu\int}}
\newcommand\tbint{{\mathchar '26\mkern -14mu\int}}
\newcommand\dbbint{{-\mkern -19mu\int}}
\newcommand\dbint{{\mathchar '26\mkern -18mu\int}}
\newcommand\bint{
{\mathchoice{\dbint}{\tbint}{\tbint}{\tbint}}
}
\newcommand\bbint{
{\mathchoice{\dbbint}{\tbbint}{\tbbint}{\tbbint}}
}
\title{Homework\# 5}
\author{Shuang Hu(26)}
\begin{document}
\maketitle
\section*{Evans}
\subsection*{P88 Problem16}
\begin{proof}
    For $u_{t}-\Delta u=0$, set $u_{\epsilon}=u-\epsilon t$, we can see:
    \begin{equation}
        \pdfFrac{u_{\epsilon}}{t}-\Delta u_{\epsilon}=u_{t}-\epsilon-\Delta u=-\epsilon<0.
    \end{equation}
    If $u_{\epsilon}$ attains its maximum in $U_{T}$, assume the maximum point is $(\mathbf{x}_{0},t_{0})$, we can see:
    \begin{equation}
        \left.
        \begin{aligned}
            \pdfFrac{u_{\epsilon}}{t}(\mathbf{x}_{0},t_{0})&=0\\
            \Delta(\mathbf{x}_{0},t_{0})&\le 0\\
        \end{aligned}
        \right\}
        \Rightarrow \pdfFrac{u_{\epsilon}}{t}-\Delta u_{\epsilon}\ge 0.
    \end{equation}
    Contradict! So $u_{\epsilon}$ gets its maximal on $\Gamma_{T}$. It means that:
    \begin{equation}
        u_{\epsilon}(\mathbf{x},t)=u-\epsilon t\le\max_{\Gamma_{T}}u.
    \end{equation}
    Set $\epsilon\rightarrow 0$, we can see $u\le\max_{\Gamma_{T}}u$ is always true.
\end{proof}
\subsection*{P88 Problem17}
(a)
\begin{proof}
    Modify the proof of theorem 3. Put $v$ instead of $u$ in the proof. We know that:
    \begin{equation}
        \phi'(r)=A+B=\frac{1}{r^{n+1}}\int_{E(r)}-4nv_{s}\psi-\frac{2n}{s}\sum_{i=1}^{n}v_{y_{i}}y_{i}\dif y\dif s.
    \end{equation}
    Since $\psi$ defined to be 
    \begin{equation}
        \psi=-\frac{n}{2}\log(-4\pi s)+\frac{|y|^{2}}{4s}+n\log(r).
    \end{equation}
    $\psi\ge 0$  in $E(r)$ because $\Phi(y,-s)r^{n}\ge1$ in $E(r)$. Thus $4n\psi(v_{s}-\Delta v)\leqslant 0$, $-4n\psi v_{s}\ge-4n\Delta v$. Then we have inequality:
    \begin{equation}
        \begin{aligned}
            \phi'(r)&=\frac{1}{r^{n+1}}\int_{E(r)}-4nv_{s}\psi-\frac{2n}{s}\sum_{i=1}^{n}v_{y_{i}}y_{i}\dif y\dif s\\
            &\ge\frac{1}{r^{n+1}}\int_{E(r)}-4n\Delta v\psi-\frac{2n}{s}\sum_{i=1}^{n}v_{y_{i}}y_{i}\dif y\dif s\\
            &=0
        \end{aligned}
    \end{equation}
    according to ghe proof of theorem 3. So we have:
    \begin{equation}
        \phi(r)\ge\phi(\epsilon)\forall r>\epsilon>0.
    \end{equation}
    But we know:
    \begin{equation}
        \lim_{\epsilon\rightarrow 0}\phi(\epsilon)=4v(0,0).
    \end{equation}
    So we have the inequality:
    \begin{equation}
        \frac{1}{r_{n}}\int_{E(r)}v(y,s)\frac{|y|^{2}}{s^{2}}\dif y\dif s=\phi(r)\ge 4v(0,0).
    \end{equation}
    WLOG, we have:
    \begin{equation}
        \frac{1}{4r^{n}}\int_{E(x,t;r)}v(y,s)\frac{|x-y|^{2}}{(t-s)^{2}}\dif y\dif s\ge v(x,t).
    \end{equation}
\end{proof}

(b)
\begin{proof}
Define a set 
\begin{equation}
    S:=\{(\mathbf{x},t):v(\mathbf{x},t)=\max_{\bar{U}_{T}}v\}.
\end{equation}
As $v$ is continuous, $S$ is a relative closed set. On the other hand, choose $(\mathbf{x},t)\in S$, by $(a)$, we can see:
\begin{equation}
    \label{eq:maximal}
    \frac{1}{4r^{n}}\int_{E(\mathbf{x},t;r)}(v_{max}-v)\frac{|x-y|^{2}}{(t-s)^{2}}\dif y\dif s\leqslant 0.
\end{equation}

As $\frac{|x-y|^{2}}{(t-s)^{2}}\ge 0$, \eqref{eq:maximal} means that $v=v_{max}$ in $E$, which means that $S$ is an open set. For $U_{T}$ is a region, we can see $S=U_{T}$.
\end{proof}

(c)
\begin{equation}
    \begin{aligned}
        v_{t}&=\phi'(u)u_{t}\\
        v_{x_{i}}&=\phi'(u)u_{x_{i}}\\
        v_{x_{i}x_{i}}&=\phi''(u)(u_{x_{i}})^{2}+\phi'(u)u_{x_{i}x_{i}}.\\
    \end{aligned}
\end{equation}
It means that:
\begin{equation}
    v_{t}-\Delta v=-\phi''(u)\sum_{i=1}^{n}(u_{x_{i}})^{2}.
\end{equation}
As $\phi$ convex, $\phi''(u)\ge 0$, which means $v_{t}-\Delta v\leqslant 0$.

(d)
\begin{equation}
    \begin{aligned}
        v_{t}&=2u_{t}u_{tt}+\add{i}{n}2u_{i}u_{it}\\
        v_{i}&=2\add{j}{n}u_{ij}u_{j}+2u_{t}u_{it}\\
        v_{ii}&=2\add{j}{n}u_{iij}u_{j}+2\add{j}{n}(u_{ij})^{2}+2(u_{it})^{2}+2u_{t}u_{iit}\\
    \end{aligned}
\end{equation}
It means:
\begin{equation}
    v_{t}-\Delta v=2u_{t}(u_{tt}-\Delta u_{t})+2\add{i}{n}\add{j}{n}u_{j}(u_{jt}-u_{iij})-2
    \add{j}{n}(u_{ij})^{2}\leqslant 0.
\end{equation}
So $v$ is a subsolution.
\section*{Modern PDE}
In the following problems, we set the notation:
\begin{equation}
    \alpha(x)=
    \left\{
        \begin{aligned}
           &e^{\frac{1}{|x|^{2}-1}},&|x|<1\\
           &0,&|x|\ge 1 
        \end{aligned}
    \right.
\end{equation}
and $\alpha_{\epsilon}(x)=\frac{1}{\epsilon^{n}}\alpha(\frac{x}{\epsilon})$.

\begin{Definition}
    The convolution of function $f$ and $g$ which is defined on $\Omega$ is:
    \begin{equation}
        f*g(x)=\int_{\Omega}f(y)g(x-y)\dif y=\int_{\Omega}f(x-y)g(y)\dif y
    \end{equation}
\end{Definition}
\subsection*{P9 Problem1}
Get $u\in C^{0}(\mathbb{R}^{n})$, set $u_{\epsilon}=u*\alpha_{\epsilon}$, by theorem 1.1, we can see:
\begin{equation}
    \label{eq:C0converge}
    \lim_{\epsilon\rightarrow 0}u_{\epsilon}=u\text{ in }C^{0}(\mathbb{R}^{n}).
\end{equation}
We just choose the sequence $f_{n}=u_{\frac{1}{n}}$, $u$ and $\alpha_{\frac{1}{n}}$ both have compact support, so $f_{n}\in C_{c}^{\infty}(\mathbb{R}^{n})$, and by \eqref{eq:C0converge}, we can see 
\begin{equation}
    f_{n}\rightrightarrows u.
\end{equation}
which means $C_{c}^{\infty}(\mathbb{R}^{n})$ is dense in $C^{0}(\mathbb{R}^{n})$.

If $u\in L^{p}(\mathbb{R}^{n})$, by Riesz theorem, $\forall \frac{1}{n}>0$, $\exists v_{n}$ s.t. $\|u-v_{n}\|_{p}<\frac{1}{n}$. By theorem 1.1, the following equation is true:
\begin{equation}
    \label{eq:Lpconverge}
    \lim_{\epsilon\rightarrow 0}v_{n\epsilon}=v_{n}(L^{p}(\mathbb{R}^{n})).
\end{equation}
Set $v_{nm}=v_{n}*\alpha_{\frac{1}{m}}$, we can see
\begin{equation}
    \lim_{m\rightarrow\infty,n\rightarrow\infty}v_{nm}=u.
\end{equation}

So $C_{c}^{\infty}$ is dense in $L^{p}(\mathbb{R}^{n})$.
\subsection*{P9 Problem2}
$\forall$ compact set $K$ and multiple index $\beta$, consider:
\begin{equation}
    \|\partial^{\beta}(J_{\epsilon}u-u)\|\leqslant\int_{\|y\|\leqslant\epsilon}\|\partial^{\beta}u(x-y)-\partial^{\beta}u(x)\|\alpha_{\epsilon}(y)\dif y.
\end{equation}
As $u\in C^{\infty}(\mathbb{R}^{n})$, we can see $\partial^{\beta}u\in C^{\infty}(\mathbb{R}^{n})$, which means $\partial^{\alpha}u$ is uniform continuous in $K$. It means:
\begin{equation}
    \forall\epsilon>0,\exists\delta,\forall|y|<\delta,|u(x-y)-u(x)|\leqslant\epsilon\Rightarrow|J_{\delta}u-u|\leqslant\epsilon.
\end{equation}
So $J_{\epsilon}u\rightarrow u(C^{\infty}(\mathbb{R}^{n})).$

If $u\in C_{c}^{\infty}(\mathbb{R}^{n})$, assume $K$ is the compact support set of $u$, then the support set of $u_{\epsilon}$ must be a subset of 
\begin{equation}
    K_{\epsilon}=\{x:\exists y\in K s.t. |x-y|\leqslant\epsilon\}.
\end{equation}
Then $\forall\epsilon\leqslant 1$, $K_\epsilon\subset K_{1}$. In the same way, in $K_{1}$, $\sup_{x\in K}|\partial^{\alpha}(u_{\epsilon}-u)|\rightarrow 0$.
\subsection*{P10 Problem3}
Mark the consistent compact support set as $K$, all the regular points in $K$ is 
\begin{equation}
    \{q_{1},q_{2},\cdots,q_{n},\cdots\}.
\end{equation}
Then $\forall$ $i$, $\{\phi_{m}(q_{i})\}$ is a Cauchy sequence in $\mathbb{R}$, so we can set 
$\phi(q_{i})=\lim_{m\rightarrow\infty}\phi_{m}(q_{i})$. Moreover, as the definition of Cauchy sequence, 
the convergence of $\{\phi_{m}(q_{i})\}$ is consistent, i.e. $\forall\epsilon>0$, $\exists N$, 
$\forall m>N$ we have $|\phi_{m}(q_{i})-\phi(q_{i})|<\epsilon$ $\forall$ $i$.

For irregular point $\mathbf{x}$ in $K$, there exists a sequence of regular points such that $\lim_{j\rightarrow\infty}q_{m_{j}}=\mathbf{x}$. 
Consider the sequence $a_{j}=\phi(q_{m_{j}})$, claim $a_{j}$ is a Cauchy sequence.

For $j_{1}\neq j_{2}$, $\forall n\in\mathbb{N}$, we have:
\begin{equation}
    |\phi(q_{m_{j_{1}}})-\phi(q_{m_{j_{2}}})|\leqslant|\phi(q_{m_{j_1}})-\phi_{n}(q_{m_{j_1}})|+|\phi_{n}(q_{m_{j_{1}}})-\phi_{n}(q_{m_{j_2}})|+|\phi(q_{m_{j_2}})-\phi_{n}(q_{m_{j_{2}}})|.
\end{equation}

As $\phi_{n}(q_{i})$ uniformly converge, and $\phi_{n}$ is continuous, we can see $\{\phi(q_{m_{i}})\}$ is a Cauchy sequence, which means that $\phi(x):=\lim_{j\rightarrow\infty}\phi(q_{m_{j}})$ is well-defined.

By such definition, we can see $\phi(x)$ is continuous.

Then, we show that $\phi_{n}\rightrightarrows\phi$ in $K$. As $\phi$ continuous in $K$, $\forall\epsilon>0$, $\exists$ $\delta_{1}(x)$, $\forall |x-x_{0}|<\delta_{1}$, $|\phi(x)-\phi(x_{0})|<\epsilon$.

As $\phi_{n}(q_{i})$ is uniformly converge, $\exists N$, $\forall n>N$, $|\phi(q_{i})-\phi_{n}(q_{i})|<\epsilon$.

As $\phi_{n}(x)$ is continuous, $\exists \delta_{2}(n,x)$ s.t. $\forall |x-x_{1}|<\delta_{2}$, $|\phi_{n}(x)-\phi_{n}(x_{1})|<\epsilon$.

Regular number is dense in $\mathbb{R}$. So $\forall x\in K$, $\exists q_{i}$ s.t. $|x-q_{i}|<\delta_{1},|x-q_{i}|<\delta_{2}$.

Summery the above cases, $\forall n>N$, we can see:
\begin{equation}
    \begin{aligned}
        |\phi(x)-\phi_{n}(x)|&\le|\phi(x)-\phi(q_{i})|+|\phi(q_{i})-\phi_{n}(q_{i})|+|\phi_{n}(q_{i})-\phi_{n}(x)|\\
        &\le 3\epsilon.
    \end{aligned}
\end{equation}

Which means that $\phi_{n}\rightrightarrows\phi$ in $K$. As $\phi_{n}\in C_{c}^{\infty}(\mathbb{R}^{n})$, we can see $\phi_{m}\rightarrow\phi(C_{c}^{\infty}(\mathbb{R}^{n}))$.
\end{document}
