\documentclass[a4paper]{ctexart}
\usepackage{geometry}
% useful packages.
\usepackage{amsfonts}
\usepackage{amsmath}
\usepackage{amssymb}
\usepackage{amsthm}
\usepackage{mathrsfs}
\usepackage{enumerate}
\usepackage{graphicx}
\usepackage{multicol}
\usepackage{fancyhdr}
\usepackage{layout}
\newtheorem{Definition}{\hspace{2em}定义}[section]
\newtheorem{Example}{\hspace{2em}例}[section]
\newtheorem{Thm}{\hspace{2em}定理}[section]
\newtheorem{Lem}{\hspace{2em}Lemma}[section]
\newtheorem{cor}{\hspace{2em}推论}[section]
\newtheorem{rem}{\hspace{2em}注记}[section]
% some common command
\newcommand{\supp}{\text{supp}}
\newcommand{\Rn}{\mathbb{R}^{n}}
\newcommand{\dif}{\mathrm{d}}
\newcommand{\avg}[1]{\left\langle #1 \right\rangle}
\newcommand{\difFrac}[2]{\frac{\dif #1}{\dif #2}}
\newcommand{\pdfFrac}[2]{\frac{\partial #1}{\partial #2}}
\newcommand{\OFL}{\mathrm{OFL}}
\newcommand{\UFL}{\mathrm{UFL}}
\newcommand{\fl}{\mathrm{fl}}
\newcommand{\op}{\odot}
\newcommand{\cp}{\cdot}
\newcommand{\Eabs}{E_{\mathrm{abs}}}
\newcommand{\Erel}{E_{\mathrm{rel}}}
\newcommand{\DR}{\mathcal{D}_{\widetilde{LN}}^{n}}
\newcommand{\add}[2]{\sum_{#1=1}^{#2}}
\newcommand{\innerprod}[2]{\left<#1,#2\right>}
\newcommand\tbbint{{-\mkern -16mu\int}}
\newcommand\tbint{{\mathchar '26\mkern -14mu\int}}
\newcommand\dbbint{{-\mkern -19mu\int}}
\newcommand\dbint{{\mathchar '26\mkern -18mu\int}}
\newcommand\bint{
{\mathchoice{\dbint}{\tbint}{\tbint}{\tbint}}
}
\newcommand\bbint{
{\mathchoice{\dbbint}{\tbbint}{\tbbint}{\tbbint}}
}
\title{Summary}
\author{Shuang Hu}
\begin{document}
\maketitle
\section{广义函数与Sobolev空间}
\begin{Example}
    考虑以下函数列:
    \begin{equation}
        \varphi_{n}(x)=
        \left\{
            \begin{aligned}
            e^{\frac{1}{n^{2}x^{2}-1}},&|x|<\frac{1}{n}\\
            0,&|x|\ge\frac{1}{n}\\
            \end{aligned}
        \right.
    \end{equation}
    当$n\rightarrow\infty$时,$\varphi_{n}(x)$在常义函数的意义下不收敛。\textbf{能否进一步扩充函数的定义,使得这样的“极限函数”存在?}
\end{Example}
\begin{Definition}[基本空间]
基本空间指满足一定条件的函数所构成的函数空间。对于区域$\Omega\subset\mathbb{R}^{n}$,本书主要讨论三个常见的基本空间:$C_{c}^{\infty}(\Omega)$,$C^{\infty}(\Omega)$,$\mathscr{S}(\Omega)$.
\end{Definition}
\begin{Definition}[$C_{c}^{\infty}(\Omega)$]
    $C_{c}^{\infty}(\Omega)$是由$\Omega$上无限次连续可微且有紧支集的函数所构成的线性空间。其上的拓扑定义如下:

    若一列函数$\varphi_{n}\rightarrow\varphi\in C_{c}^{\infty}(\Omega)$, 则这列函数满足下面两个条件:
    \begin{itemize}
        \item $\cup_{n=1}^{\infty}\supp(\varphi_{n})\subset K$, $K$是$\mathbb{R}^{n}$中的紧集。
        \item \begin{equation}
            \|\partial^{\alpha}\varphi_{n}(x)-\partial^{\alpha}\varphi(x)\| \rightrightarrows0 \forall \alpha\in\mathbb{Z}^{n}.
        \end{equation}
    \end{itemize}
\end{Definition}
\begin{Definition}[$C^{\infty}(\Omega)$]
    $C^{\infty}(\Omega)$是由$\Omega$上无限次连续可微且在无穷远处趋于0的函数所构成的线性空间。其上的拓扑定义如下:
    
    若一列函数$\varphi_{n}\rightarrow\varphi\in C^{\infty}(\Omega)$,则对任意紧集$K$和多重指标$\alpha$,
    \begin{equation}
        \|\partial^{\alpha}\varphi_{n}(x)-\partial^{\alpha}\varphi(x)\| \rightrightarrows0.(x\in K).
    \end{equation}
\end{Definition}
\begin{Definition}[$\mathscr{S}(\Omega)$]
    $\mathscr{S}(\Omega)$由满足如下条件的函数组成:
    \begin{equation}
        \lim_{|x|\rightarrow\infty}x^{\alpha}\partial^{\beta}\varphi(x)\rightarrow 0\forall\alpha,\beta.
    \end{equation}    
    其上的拓扑:

    若$\varphi_{n}\rightarrow 0$, 则
    \begin{equation}
        \sup_{x\in\mathbb{R}^{n}}x^{\alpha}\partial^{\beta}\varphi_{n}(x)=0.
    \end{equation}
\end{Definition}
\begin{Thm}
    \begin{equation}
        C_{c}^{\infty}(\Omega)\subset\mathscr{S}(\Omega)\subset C^{\infty}(\Omega).
    \end{equation}
\end{Thm}
\begin{Thm}
    $C_{c}^{\infty}(\mathbb{R}^{n})$在$L^{p}(\mathbb{R}^{n})$和$C^{0}(\mathbb{R}^{n})$中稠密。
\end{Thm}
证明提示:$L^{p}$函数定义,Lusin定理,利用卷积实现光滑化。
\begin{Definition}
    定义
    \begin{equation}
        \varphi(x)=
        \left\{
            \begin{aligned}
            e^{\frac{1}{|x|^{2}-1}},&|x|<1\\
            0,&|x|\ge1\\
            \end{aligned}
        \right.
    \end{equation}
    由此导出光滑化子$\alpha_{\epsilon}:=\frac{1}{\epsilon^{n}}\varphi(\frac{x}{\epsilon})$。该函数满足两个条件:
    \begin{itemize}
        \item $\alpha_{\epsilon}\in C_{c}^{\infty}(\mathbb{R}^{n})$.
        \item $\int_{\mathbb{R}^{n}}\alpha_{\epsilon}(x)\dif x=1$.
    \end{itemize}
\end{Definition}
\begin{Definition}[局部可积]
    如果$\Omega$上的一个函数$\varphi$在任意紧集$K\subset\Omega$上Lebesgue可积,则称该函数在$\Omega$上\textbf{局部可积},记$\varphi\in L_{loc}^{1}(\Omega)$.
\end{Definition}
\begin{Definition}
    设$u\in L_{loc}^{1}(\mathbb{R}^{n})$, 则$u_{\epsilon}(x):=u*\alpha_{\epsilon}\in C^{\infty}(\mathbb{R}^{n})$. 当$\epsilon\rightarrow 0$, 若$u\in X$, $X=C^{0}(\mathbb{R}^{n})$或$X=L^{p}(\mathbb{R}^{n})$, 则$u_{\epsilon}\rightarrow u(X)$.
\end{Definition}
\begin{Definition}
    对于基本空间$X$,$X$上的\textbf{有界线性泛函}称为$X$上的\textbf{广义函数}。如,我们称$C_{c}^{\infty}(\mathbb{R}^{n})$上的有界线性泛函为$\mathcal{D}'(\mathbb{R}^{n})$广义函数。
\end{Definition}
\begin{Thm}
    \begin{equation}
        \mathcal{\epsilon}'(\mathbb{R}^{n})\subset\mathscr{S}'(\mathbb{R}^{n})\subset\mathcal{D}'(\mathbb{R}^{n})
    \end{equation}
\end{Thm}
\begin{Thm}
    \begin{equation}
    L_{loc}^{1}(\Omega)\subset\mathcal{D}'(\Omega).
    \end{equation}
    对应$f\in L_{loc}^{1}(\Omega)$的泛函为:
    \begin{equation}
        \innerprod{f}{\varphi}:=\int_{\Omega}f\varphi\dif x.
    \end{equation}
\end{Thm}
\begin{Definition}
    定义基本函数空间$X$上的泛函$\delta$为:对$\phi\in X$,
    \begin{equation}
        \delta(\phi):=\phi(0).
    \end{equation}
\end{Definition}
\begin{Thm}
    $\delta\notin L_{loc}^{1}(\Omega)$.
\end{Thm}
\begin{Thm}
    $\delta\in\mathcal{D}'(\Omega)$,$\delta\in\mathscr{S}'(\Omega)$,$\delta\in\epsilon'(\Omega)$.
\end{Thm}
\begin{Definition}[支集]
    \label{support}
    函数$\varphi\in C^{\infty}(\Omega)$的支集定义为:
    \begin{equation}
        \supp\varphi:=\mathcal{C}(\{\mathbf{x}:\varphi(\mathbf{x})\neq 0\})
    \end{equation}
    $\mathcal{C}$表示取闭包。
\end{Definition}
\begin{rem}
    一般我们无法按\ref{support}的形式定义一个分布的支集,因为分布是整体定义的,我们一般无法讨论一个分布"在某点处"的取值。
\end{rem}
\begin{Definition}
    对于$T\in\mathcal{D}'(\Omega)$, 如果对任意的$\varphi\in C_{c}^{\infty}(\Omega')$, 有
    \begin{equation}
        \innerprod{T}{\varphi}=0,
    \end{equation}
    则$T$在$\Omega'$内\textbf{取零值}。
\end{Definition}
\begin{Thm}
    如果广义函数$T\in\mathcal{D}'(\Omega)$在每个点的开邻域上取零值,那么在整个$\Omega$上$T\equiv 0$。
\end{Thm}
证明提示:有限覆盖+单位分解。
\begin{Definition}
    使广义函数$T$取零值的最大开集的补集,称广义函数$T$的\textbf{支集},记为$\supp T$。
\end{Definition}
\begin{Example}
    $\supp\delta=\{0\}$.
\end{Example}
\begin{Thm}
    任何$\epsilon'(\Omega)$的广义函数$T$具有紧支集。任意具有紧支集的$T\in\mathcal{D}'(\Omega)$必为$\epsilon'(\Omega)$广义函数。
\end{Thm}
\begin{Definition}[弱极限]
    如果一列属于某一基本函数空间的广义函数列$\{T_{k}\}$,对基本空间上的任意元素$\varphi$,满足:
    \begin{equation}
        \innerprod{T_{k}}{\varphi}\rightarrow 0.
    \end{equation}
    则称$T_{k}$\textbf{弱收敛}于0。
\end{Definition}
\begin{Example}
    \begin{equation}
        \lim_{\epsilon\rightarrow 0} \frac{2x}{x^2+\epsilon^2}=P.V.(\frac{2}{x}).
    \end{equation}
\end{Example}
\begin{Thm}
    若$T\in\mathcal{D}'(\mathbb{R}_{y}^{n})$,则$\innerprod{T_{y}}{\alpha_{\epsilon}(x-y)}$为$\mathbb{R}_{x}^{n}$上的$C^{\infty}$函数。

    这一过程称为\textbf{广义函数$T$的正则化}。
\end{Thm}
\begin{Thm}
    当$T\in\mathcal{D}'(\Rn)$,$T_{\epsilon}\rightarrow T(\mathcal{D}'(\Rn))$.
\end{Thm}
证明思路:$T_{\epsilon}$为常义函数,其作用在$\varphi$上的值可用积分表示。估计积分表达式即可。
\begin{Definition}[广义导数]
    广义函数$T$的$\alpha$阶导数定义为:
    \begin{equation}
        \innerprod{\partial^{\alpha}T}{\varphi}=(-1)^{|\alpha|}\innerprod{T}{\partial^{\alpha}\varphi}.
    \end{equation}
\end{Definition}
\begin{Example}
    Heaviside函数$H(x)$满足$H'(x)=\delta$.
\end{Example}
\begin{Thm}
    如果一个广义函数的导数为连续函数,那么这个广义函数一定是常义的$C^{1}$函数。
\end{Thm}
证明思路:$-\innerprod{T}{\partial\varphi}=\innerprod{g}{\varphi}$,考察$\int g\dif x$.
\begin{Definition}
    设$T\in\mathcal{D}'(\Rn)$,$\alpha\in C^{\infty}(\Rn)$, 定义:
    \begin{equation}
        \innerprod{\alpha T}{\varphi}=\innerprod{T}{\alpha\varphi}.
    \end{equation}
\end{Definition}
\begin{Definition}
    设$S,T$为两个广义函数,则$S*T$定义为:
    \begin{equation}
        \innerprod{S*T}{\varphi}=\innerprod{S_{x}}{\innerprod{T_{y}}{\varphi(x+y)}}.
    \end{equation}
\end{Definition}
\begin{Example}
    \begin{equation}
        tH(t)*e^{t}H(t)=(e^{t}-t-1)H(t).
    \end{equation}
\end{Example}
\begin{Thm}
    如果$T$为$\epsilon'(\Rn)$广义函数,$S$为$\mathcal{D}'(\Rn)$广义函数,那么$S*T$为$\mathcal{D}'(\Rn)$广义函数。
\end{Thm}
证明思路:设$T$的紧支集为$K$,构造辅助函数$\zeta(y)$使得其在$K$上取值为1。
\begin{Thm}
    $\epsilon'(\Rn)$广义函数全体关于线性运算和卷积运算构成一个可交换的有单位元的代数,$\delta$为其单位元。
\end{Thm}
\begin{Definition}[$\mathscr{S}(\Rn)$上的Fourier变换]
    对于函数$f\in\mathscr{S}(\mathbb{R}^{n})$,定义其Fourier变换为:
    \begin{equation}
        F[f]:=\int_{\mathbb{R}^{n}}f(x)e^{-ix\cdot\xi}\dif x.
    \end{equation}
    又若$g(\xi)\in\mathscr{S}(\mathbb{R}^{n})$,定义其Fourier逆变换为:
    \begin{equation}
        F^{-1}(g)=\frac{1}{(2\pi)^{n}}\int_{\mathbb{R}^{n}}g(\xi)e^{i\xi\cdot x}\dif\xi.
    \end{equation}
\end{Definition}
\begin{Thm}
    \begin{equation}
        F^{-1}(F[f])=f.
    \end{equation}
\end{Thm}
\begin{Example}
    \begin{equation}
        F[\frac{1}{a^2+x^2}]=\frac{\pi}{a}e^{-a|\xi|}
    \end{equation}
\end{Example}
\begin{Thm}[Fourier变换的基本性质]
    $\newline$
    \begin{itemize}
        \item 线性:
        \begin{equation}
            F[\alpha_{1}f_{1}+\alpha_{2}f_{2}]=\alpha_{1}F[f_{1}]+\alpha_{2}F[f_{2}].
        \end{equation}
        \item 求导:
        \begin{equation}
            F[\partial_{j}f]=i\xi_{j}F[f].
        \end{equation}
        \item 乘多项式
        \begin{equation}
            F[x_{j}f]=i\partial_{j}F[f].
        \end{equation}
        \item 卷积
        \begin{equation}
            \begin{aligned}
            F[f*g]&=F[f].F[g]\\
            F[f.g]&=(2\pi)^{-n}F[f]*F[g]\\
            \end{aligned}
        \end{equation}
    \end{itemize}
\end{Thm}
\begin{Thm}[Parseval]
    \begin{equation}
        \int_{\Rn}f\cdot\bar{g}\dif x=(2\pi)^{-n}\int_{\Rn}F[f]\overline{F[g]}\dif x.
    \end{equation}
\end{Thm}
\begin{Definition}[$\mathscr{S}'(\Rn)$上的Fourier变换]
    对于任意$\mathscr{S}'(\Rn)$广义函数$T$,定义其Fourier变换$F[T]$为:
    \begin{equation}
        \innerprod{F[T]}{\varphi}=\innerprod{T}{F[\varphi]}.
    \end{equation}
    逆Fourier变换同样定义。
\end{Definition}
广义函数Fourier变换的性质大多和常义函数相同,但卷积性质略有区别,因为广义函数的卷积不一定可以定义。
\begin{Thm}
    若$\varphi\in\mathscr{\Rn}$,$T\in\mathscr{S}'(\Rn)$,则$\varphi*T\in\mathscr{S}'(\Rn)$且
    \begin{equation}
        F[\varphi*T]=F[\varphi]F[T].
    \end{equation}
\end{Thm}
\begin{Thm}
    对于$S\in\mathscr{S}'(\Rn),T\in\epsilon'(\Rn)$,有:
    \begin{equation}
        F[T*S]=F[T].F[S]
    \end{equation}
\end{Thm}
\begin{Example}
    \begin{equation}
        F[\delta(x)]=1.
    \end{equation}
\end{Example}
\begin{Example}
    \begin{equation}
        F[1]=(2\pi)^{n}\delta.
    \end{equation}
\end{Example}
\begin{Example}
    \begin{equation}
        F[\sin ax]=-i\pi(\delta(\xi-a)-\delta(\xi+a)).
    \end{equation}
\end{Example}
设在空间$\Rn$中给定线性微分算子
\begin{equation}
    p(x,D)=\sum_{|\alpha|\le M}a_{\alpha}(x)D^{\alpha},D=\frac{1}{i}\partial.
\end{equation}
则由Fourier变换的性质:
\begin{equation}
    p(x,D)u=\frac{1}{(2\pi)^{n}}\int e^{i\innerprod{x}{\xi}}p(x,\xi)\hat{u}(\xi)\dif\xi.
\end{equation}
\begin{Definition}[微分算子的推广]
    如果$a(x,\xi)\in S^{m}$,可以定义$\mathscr{S}(\Rn)$上的线性连续映射为:
    \begin{equation}
        Au(x)=\frac{1}{(2\pi)^{n}}\int e^{i\innerprod{x}{\xi}}a(x,\xi)\hat{u}(\xi)\dif\xi
    \end{equation}
    $a(x,\xi)$称为$A$的\textbf{象征}。
\end{Definition}
\begin{Definition}[非负整指数Sobolev空间]
    设$\Omega\subset\Rn$是一个给定区域,对$m\ge 0,1\le p\le \infty$,可以定义Sobolev空间$W^{m,p}(\Omega)$为满足条件
    \begin{equation}
        D^{\alpha}u\in L^{p}(\Omega)
    \end{equation}
    的所有广义函数$u$构成的集合,定义其上的范数为:
    \begin{equation}
        \label{norm}
        \|u\|:=(\sum_{|\alpha|\le m}\|D^{\alpha}u\|_{L^{p}(\Omega)}^{p})^{\frac{1}{p}}.
    \end{equation}
    在$p=2$时,记$W^{m,2}(\Omega):=H^{m}(\Omega)$。此时可以在这个线性空间上赋予内积如下:
    \begin{equation}
        \innerprod{u}{v}_{m}:=\sum_{|\alpha|\le m}\innerprod{D^{\alpha}u}{D^{\alpha}v}_{L^{2}(\Omega)}.
    \end{equation}
\end{Definition}
\begin{Thm}
    $W^{m,p}(\Omega)$是Banach空间。
\end{Thm}
提示:利用$L^{p}$空间的完备性。

\begin{Thm}
    设$\Omega$是边界光滑的有界区域,则任意$W^{m,p}(\Omega)$中的函数$u$都可以用$C^{\infty}(\overline{\Omega})$函数来逼近。此即,$W^{m,p}(\Omega)$可以看作在$C^{\infty}$中赋予范数\eqref{norm}后的完备化空间。该空间也记为$H^{m,p}(\Omega)$。
\end{Thm}
证明思路:局部化+展平(太长了,也看不懂)
\begin{Thm}
    $m\ge 1$时,$C_{c}^{\infty}(\Rn_{+})$在$H^{m,p}(\Rn_{+})$上并不稠密。
\end{Thm}
提示:考虑$v(x)=e^{-|x|^{2}}$.
\begin{Definition}
    $C_{c}^{\infty}(\Omega)$函数按\eqref{norm}完备化形成的空间记为$H_{0}^{m,p}(\Omega)$。
\end{Definition}
\begin{Thm}
    $\Omega$有界且$m\ge 1$时,$H_{0}^{m,p}(\Omega)\subsetneqq H^{m,p}(\Omega)$。
\end{Thm}
因为这个时候$u|_{\partial\Omega}\equiv 0$。
\begin{Definition}[负整指数Sobolev空间]
    对正整数$m$,$1\le p<\infty$,将$H_{0}^{m,p}(\Omega)$的对偶空间$(H_{0}^{m,p}(\Omega))'$\textbf{视为$\mathcal{D}'(\Omega)$的子空间},称为$H^{-m,p'}(\Omega)$,其中$\frac{1}{p}+\frac{1}{p'}=1$。
\end{Definition}
\begin{rem}
    我们把$H^{-m}(\Omega)$视作$\mathcal{D}'(\Omega)$的子空间,则对应的泛函应当是:
    \begin{equation}
        v(u)=\innerprod{v}{\bar{u}}_{L^{2}(\Omega)}
    \end{equation}
    而不是:
    \begin{equation}
        v(u)=\sum_{|\alpha|\le m}\innerprod{D^{\alpha}v}{D^{\alpha}u}_{L^{2}(\Omega)}.
    \end{equation}
\end{rem}
在负整指数Sobolev空间中,可以引入如下范数(实际上就是算子范数):
\begin{equation}
    \|u\|_{H^{-m,p'}(\Omega)}=\sup_{v\in H_{0}^{m,p}(\Omega)}\frac{|\innerprod{u}{v}|}{\|v\|_{H_{0}^{m,p}(\Omega)}}.
\end{equation}
当$s$是一般的实数时,我们可以利用Fourier变换给出$H^{s}(\Rn)$的定义。
\begin{Definition}
    对于$s\in\mathbb{R}$,记$H^{s}(\Rn)$是使得
    \begin{equation}
        (1+|\xi|^{2})^{\frac{s}{2}}\hat{u}(\xi)\in L^{2}(\Rn)
    \end{equation}
    的所有广义函数$u\in\mathscr{S}'(\Rn)$构成的空间。其上内积定义为:
    \begin{equation}
        \innerprod{u}{v}_{s}:=\int_{\Rn}(1+|\xi|^{2})^{s}\hat{u}(\xi)\overline{\hat{v}(\xi)}\dif\xi.
    \end{equation}
\end{Definition}
\begin{Thm}
    $H^{s}(\Rn)$是Hilbert空间,且$C_{c}^{\infty}(\Rn)$在$H^{s}(\Rn)$中稠密。
\end{Thm}
\begin{Example}
    $K=I-\varDelta$,$f\in H^{-\infty}(\Rn)$, $(I-\varDelta)^{m}f\in H^{t}(\Rn)$, 则$f\in H^{2m+t}(\Rn)$。
\end{Example}
\begin{Thm}[Sobolev不等式]
    设$p$满足$1\ge \frac{1}{p}>\frac{k}{n}$,$\frac{1}{q}=\frac{1}{p}-\frac{k}{n}$,$k\le m$,则对于$u\in C_{c}^{\infty}(\Rn)$,有:
    \begin{equation}
        \|u\|_{H^{m-k,q}(\Rn)}\le C\|u\|_{H^{m,p}(\Rn)}.
    \end{equation}
\end{Thm}
\begin{Thm}[Sobolev嵌入定理]
    以下,$\frac{1}{q}=\frac{1}{p}-\frac{k}{n}$。
    
    (1)若$1\ge\frac{1}{p}>\frac{k}{n}$,$k\le m$,则$H^{m,p}(\Omega)$可嵌入$H^{m-k,q}(\Omega)$。

    (2)若$\frac{1}{p}=\frac{k}{n},k\le m$,则对任意$q\in\mathbb{R}$,$H^{m,p}(\Omega)$可以嵌入$H^{m-k,q}(\Omega)$。

    (3)设$s>\frac{n}{2}+k$,则$H^{s}(\Rn)$可以连续嵌入到$C^{k}(\Rn)$。
\end{Thm}
\begin{Thm}[迹定理]
    设$\gamma$是$C^{\infty}(\overline{\Rn_{+}})$函数$\varphi(x_{1},\cdots,x_{n})$到边界$x_{n}=0$上取值的映射,则它可以连续扩张到整个$H^{1}(\Rn_{+})$上,且$\gamma(H^{1}(\Rn_{+}))
    \subset H^{\frac{1}{2}}(\mathbb{R}^{n-1})$。
\end{Thm}
\section{偏微分方程一般理论}
\begin{Definition}[特征]
    考察线性方程
    \begin{equation}
        \label{LinearPDE}
        \sum_{|\alpha|\le m}a_{\alpha}(x)\partial_{x}^{\alpha}u=f(x).
    \end{equation}
    其线性主部为:
    \begin{equation}
        \sum_{|\alpha|=m}a_{\alpha}(x)\partial_{x}^{\alpha}u.
    \end{equation}
    如果曲面$\pi:\varphi(x_{1},\cdots,x_{n})=0$满足:
    \begin{equation}
        \sum_{|\alpha|=m}a_{\alpha_{1}\cdots\alpha_{n}}(x)(\varphi_{x_{1}})^{\alpha_{1}}\cdots(\varphi_{x_{n}})^{\alpha_{n}}=0.
    \end{equation}
    则称$\pi$为方程\eqref{LinearPDE}的\textbf{特征曲面}。
\end{Definition}
根据上一章的理论,方程\eqref{LinearPDE}的微分算子对应的象征为:
\begin{equation}
    \sum_{|\alpha|\le m}a_{\alpha}(x)(i\xi)^{\alpha}.
\end{equation}
\begin{Definition}[特征集]
    方程\eqref{LinearPDE}的特征集为:
    \begin{equation}
        \varGamma:=\{(x,\xi)|\xi\neq 0,\sum_{|\alpha|=m}a_{\alpha}(x)\xi^{\alpha}=0\}.
    \end{equation}
\end{Definition}
\begin{Definition}[主象征]
    \eqref{LinearPDE}对应的主象征定义为:
    \begin{equation}
        p_{m}(x,\xi)=\sum_{|\alpha|=m}a_{\alpha}(x)(i\xi)^{\alpha}.
    \end{equation}
\end{Definition}
接下来我们需要利用主象征函数对偏微分方程进行分类。
\begin{Definition}[椭圆算子]
    若对某一点$x\in\Omega$以及任意的$\xi\in\Rn\setminus\{0\}$,$p_{m}(x,\xi)\neq 0$,则称算子$p(x,D)$在$x$点为\textbf{椭圆型算子}。
\end{Definition}
\begin{Example}
    Laplace算子$\varDelta$的主象征是:
    \begin{equation}
        p_{2}(x,\xi)=-\sum_{i=1}^{n}\xi_{i}^{2}.
    \end{equation}
    该函数为0当且仅当$\xi=0$,所以该算子是椭圆型的。
\end{Example}
\begin{Definition}[双曲型算子]
    如果存在一个方向$\tau\in\Rn\setminus\{0\}$,使得对一切$x\in\Omega$以及任意不平行于$\tau$的方向$\xi$,方程$p_{m}(x,\lambda\tau+\xi)=0$关于$\lambda$有$m$个两两互异实根,则称$p(x,D)$为关于方向$\tau$的严格双曲型算子。
\end{Definition}
\begin{Example}
    PDE$\pdfFrac{^2u}{t^2}=a^2(\pdfFrac{^2u}{x^2}+\pdfFrac{^2u}{y^2})$在特征锥$t^2>a^2(x^2+y^2)$上严格双曲。
\end{Example}
\begin{Definition}[主型算子]
    如果在$P(x,D)$的特征集上$\nabla_{\xi}p_{m}(x,\xi)\neq 0$,则称$P(x,D)$为\textbf{主型算子}。
\end{Definition}
\begin{Definition}[抛物型算子]
    在$\mathbb{R}^{n+1}$中形为:
    \begin{equation}
        \partial_{x_{0}}-\sum_{j,k=1}^{n}a_{jk}(x)\partial_{x_{j}}\partial_{x_{k}}+\sum_{j=1}^{n}b_{j}(x)\partial_{x_{j}}+c(x)
    \end{equation}
    的二阶偏微分算子,在矩阵$(a_{jk})$正定时,称为\textbf{抛物型算子}。
\end{Definition}

\begin{Definition}[基本解]
    对于一个线性微分算子$P(\partial)$,如果存在广义函数$E$使得$P(\partial)E=\delta$,则称$E$为算子$P(\partial)$的\textbf{基本解}。
\end{Definition}
下面用两个例子介绍求基本解的两种常见方法。
\begin{Example}[首次积分法]
    求方程
    \begin{equation}
        \difFrac{y}{x}+ay=0
    \end{equation}
    的基本解。
\end{Example}
我们知道,$\difFrac{ye^{ax}}{x}=e^{ax}(\difFrac{y}{x}+ay)$。那么,要求
\begin{equation}
    \difFrac{y}{x}+ay=\delta
\end{equation}
的解,可以将其转化为:
\begin{equation}
    \difFrac{}{x}(ye^{ax})=e^{ax}\delta=\delta.
\end{equation}
从而:
\begin{equation}
    ye^{ax}=H(x)\Rightarrow y=e^{-ax}H(x).
\end{equation}
\begin{Example}[Fourier变换法]
    求Cauchy-Riemann算子$\pdfFrac{}{x}+i\pdfFrac{}{y}$的基本解。
\end{Example}
在用FT法求基本解时,主要的依据是$F[\delta]=1$.

如果广义函数$T$满足:
\begin{equation}
    \pdfFrac{T}{x}+i\pdfFrac{T}{y}=\delta(x,y),
\end{equation}
两边关于$y$作Fourier变换得到:
\begin{equation}
    \pdfFrac{\hat{T}}{x}-\eta\hat{T}=\delta(x).
\end{equation}
解得:
\begin{equation}
    \label{hatT}
    \hat{T}=e^{\eta x}H(x)+C(\eta)e^{\eta x}.
\end{equation}
要使$\hat{T}\in\mathscr{S}(\Rn)$,需要取$C(\eta)=-H(\eta)$。代入\eqref{hatT},作逆Fourier变换即可得$T(x,y)=\frac{1}{2\pi}\frac{1}{x+iy}$.
\begin{Definition}[Cauchy问题的基本解]
    考虑Cauchy问题:
    \begin{equation}
        \begin{aligned}
            &\pdfFrac{u}{t}=P(\pdfFrac{}{x})u\\
            &u(0,x)=u_{0}(x)\\
        \end{aligned}
    \end{equation}
    该方程满足初始条件$u(0,x)=\delta(x)$的解称为基本解。
\end{Definition}
\end{document}