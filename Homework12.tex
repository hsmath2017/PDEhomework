\documentclass[a4paper]{ctexart}
\usepackage{geometry}
% useful packages.
\usepackage{amsfonts}
\usepackage{amsmath}
\usepackage{amssymb}
\usepackage{amsthm}
\usepackage{mathrsfs}
\usepackage{enumerate}
\usepackage{graphicx}
\usepackage{multicol}
\usepackage{fancyhdr}
\usepackage{layout}
\newtheorem{Definition}{\hspace{2em}定义}
\newtheorem{Example}{\hspace{2em}例}
\newtheorem{Thm}{\hspace{2em}定理}
\newtheorem{Lem}{\hspace{2em}引理}
\newtheorem{cor}{\hspace{2em}推论}
% some common command
\newcommand{\dif}{\mathrm{d}}
\newcommand{\avg}[1]{\left\langle #1 \right\rangle}
\newcommand{\difFrac}[2]{\frac{\dif #1}{\dif #2}}
\newcommand{\pdfFrac}[2]{\frac{\partial #1}{\partial #2}}
\newcommand{\OFL}{\mathrm{OFL}}
\newcommand{\UFL}{\mathrm{UFL}}
\newcommand{\fl}{\mathrm{fl}}
\newcommand{\op}{\odot}
\newcommand{\cp}{\cdot}
\newcommand{\Eabs}{E_{\mathrm{abs}}}
\newcommand{\Erel}{E_{\mathrm{rel}}}
\newcommand{\DR}{\mathcal{D}_{\widetilde{LN}}^{n}}
\newcommand{\add}[2]{\sum_{#1=1}^{#2}}
\newcommand{\innerprod}[2]{\left<#1,#2\right>}
\newcommand\tbbint{{-\mkern -16mu\int}}
\newcommand\tbint{{\mathchar '26\mkern -14mu\int}}
\newcommand\dbbint{{-\mkern -19mu\int}}
\newcommand\dbint{{\mathchar '26\mkern -18mu\int}}
\newcommand\bint{
{\mathchoice{\dbint}{\tbint}{\tbint}{\tbint}}
}
\newcommand\bbint{
{\mathchoice{\dbbint}{\tbbint}{\tbbint}{\tbbint}}
}
\title{Homework\# 12}
\author{Shuang Hu(26)}
\begin{document}
\maketitle
\section*{P111 Problem2}
\begin{proof}
    Now the equation is:
    \begin{equation}
        \label{eq:Poisson3rd}
        \left\{
            \begin{aligned}
                Lu&=f,x\in\Omega\\
                \pdfFrac{u}{\nu}+\sigma u&=0,x\in\partial\Omega\\
            \end{aligned}
        \right.
    \end{equation}
    By $P110$ equation $(1.13),(1.14),(1.15)$, we can see: if $u\in H^{2}(\Omega)$, for $Lu=f$, $\forall v\in H^{1}(\Omega)$, the following equation is true:
    \begin{equation}
        \label{eq:Generalize1}
        (-f,v)_{L^{2}(\Omega)}=a(u,v)-\left(\pdfFrac{u}{\nu},v\right)_{L^{2}(\partial\Omega)}.
    \end{equation}
    By condition:
    \begin{equation}
        \label{eq:trialFunc}
        a(u,v)+(\sigma u,v)_{L^{2}(\partial\Omega)}=-(f,v),\forall v\in H^{1}(\Omega)
    \end{equation}
    \eqref{eq:Generalize1} and \eqref{eq:trialFunc} shows that:
    \begin{equation}
        \label{eq:Trace}
        \left(\pdfFrac{u}{\nu}+\sigma u,v\right)_{L^2(\partial\Omega)}=0\forall v\in H^{1}(\Omega).
    \end{equation}
    Assume $T$ is the trace map from $H^{1}(\Omega)$ to $L^{2}(\partial\Omega)$, by \eqref{eq:Trace}, it means that:
    \begin{equation}
        \label{eq:Trace2}
        \left(\left(\pdfFrac{u}{\nu}+\sigma u\right)|_{\partial\Omega},Tv\right)_{L^2(\partial\Omega)}=0\forall v\in H^{1}(\Omega).
    \end{equation}
    \eqref{eq:Trace2} shows that $\forall \varphi\in L^{2}(\partial\Omega)$,
    \begin{equation}
        \label{eq:Trace3}
        \left(\left(\pdfFrac{u}{\nu}+\sigma u\right)|_{\partial\Omega},\varphi\right)_{L^2(\partial\Omega)}=0.
    \end{equation}
    It means that $\pdfFrac{u}{\nu}+\sigma u=0$ on $\partial \Omega$ in the meaning of trace.
\end{proof}
\section*{P111 Problem3}
By $(1.18)$, the generalized solution $u\in H^{1}(\Omega)$ of this equation satisfies:
\begin{equation}
    \label{eq:GeneralizedSol}
    \int_{\Omega}\nabla u\cdot\nabla v\dif x+\sigma\int_{\partial\Omega}uv\dif S=-\int_{\Omega}fv\dif x, \forall v\in H^{1}(\Omega).
\end{equation}
We set the functional
\begin{equation}
    \label{eq:FunctionalFor3rdBVP}
    J[u]=\frac{1}{2}\sigma\int_{\partial\Omega}u^{2}\dif S+\int_{\Omega}(\frac{1}{2}|Du|^2+uf)\dif x.
\end{equation}
Then $\forall v\in H^{1}(\Omega)$, we can see:
\begin{equation}
    \label{eq:Dir}
    \begin{aligned}
        \frac{J[u+\epsilon v]-J[u]}{\epsilon}&\rightarrow    \int_{\Omega}\nabla u\cdot\nabla v\dif x+\sigma\int_{\partial\Omega}uv\dif S+\int_{\Omega}fv\dif x
    \end{aligned}
\end{equation}
when $\epsilon\rightarrow 0$. From \eqref{eq:GeneralizedSol} and \eqref{eq:Dir}, we can see \eqref{eq:FunctionalFor3rdBVP} is the functional for the 3rd BVP equation.
\section*{P111 Problem4}
The functional related to a homogenious Dirichlet problem for Laplacian equation is
\begin{equation}
    J[u]=\int_{\Omega}(\frac{1}{2}|Du|^2)\dif x.
\end{equation}
Now we should consider the target function $u-f\in H^{1}(B)$, by $(1.9),(1.10),(1.11)$, it is suffices to find a generalized solution for the Laplacian equation
\begin{equation}
    \left\{
        \begin{aligned}
            \Delta u&=0(x\in\Omega)\\
            u&=f(x\in\partial\Omega)\\
        \end{aligned}
    \right.
\end{equation}
By Poisson's formula, the solution is:
\begin{equation}
    \label{eq:Poisson}
    u(x)=\frac{1-|x|^2}{2\pi}\int_{\partial B(0,1)}\frac{f(y)}{|x-y|^{2}}\dif S(y)
\end{equation} 
And, by $\Delta u=0$ in $\Omega$, use Gauss-Green formula, we can see:
\begin{equation}
    \label{eq:GaussGreen}
    \int_{B}|\nabla u|^{2}\dif x=\int_{B}\nabla\cdot(u\nabla u)\dif x=\int_{\partial B}u\pdfFrac{u}{n}\dif S(x).
\end{equation}
By \eqref{eq:Poisson} and \eqref{eq:GaussGreen}, the result is $\pi$.
\section*{P124 Problem2}
\begin{proof}
    Extend the boundary condition $g\in H^{\frac{1}{2}}(\partial\Omega)$ to the region $\Omega$, by the extension theorem, we can extend it to $G\in H^{1}(\Omega)$, and $G|_{\partial\Omega}=g$. Set $w=u-G$, we can see:
    \begin{equation}
        \label{eq:auxiliaryPDE}
        \left\{
            \begin{aligned}
                Lw&=f-LG,x\in\Omega\\
                w&=0,x\in\partial\Omega\\
            \end{aligned}
        \right.
    \end{equation}
    Then use theorem 2.2, $\exists\lambda$ such that:
    \begin{equation}
        \label{eq:BasicIneq2}
        \|-Lw+\lambda w\|_{-1}\ge C\|w\|_{1}.
    \end{equation}
    By \eqref{eq:BasicIneq2},\eqref{eq:auxiliaryPDE} and triangular inequality, we can see:
    \begin{equation}
        \label{eq:estimate1}
        \|f-LG\|_{-1}+|\lambda|\|u-G\|_{-1}\ge C\|u-G\|_{1}\ge C\|u\|_{1}-C\|G\|_{1}.
    \end{equation}
    After the extension, $\|G\|_{1}\le C_{2}\|g\|_{\frac{1}{2}}$, $\|G\|_{-1}\le C_{3}\|g\|_{\frac{1}{2}}$by \eqref{eq:estimate1}, we can see:
    \begin{equation}
        C\|u\|_{1}\le CC_{2}\|g\|_{\frac{1}{2}}+\|f\|_{-1}+\|LG\|_{-1}+\|\lambda u\|_{-1}+\|\lambda C_{3}g\|_{\frac{1}{2}}.
    \end{equation}
    Finally, as $L$ is eliptic, $\|LG\|_{-1}\le C_{4}\|G\|_{-1}$. Above all, we can see: $\exists C>0$ such that:
    \begin{equation}
        \|u\|_{H^{1}(\Omega)}\le C(\|f\|_{H^{-1}(\Omega)}+\|g\|_{H^{\frac{1}{2}}(\partial\Omega)}+\|u\|_{H^{-1}(\Omega)}).
    \end{equation}
\end{proof}
\section*{P124 Problem3}
Assume the result is false, i.e. for $\lambda\notin\Lambda $, $\exists$ function sequence $\{u_{n}\}$ such that:
\begin{itemize}
    \item $\|u_{n}\|_{1}=1$.
    \item $\|-Lu_{n}+\lambda u_{n}\|_{-1}<\frac{1}{n}\|u_{n}\|_{1}$.
\end{itemize}
$u_{n}\in C_{c}^{\infty}(\Omega)$, $C_{c}^{\infty}(\Omega)$ is a complete space, it means that $\exists u_{n_{i}}\rightarrow u_{0}$ in $C_{c}^{\infty}(\Omega)$. Then on the second condition, we can see:
\begin{equation}
\|-Lu_{n_{i}}+\lambda u_{n_{i}}\|_{-1}<\frac{1}{n_{i}}\|u_{n_{i}}\|_{1}.
\end{equation}
Set $i\rightarrow\infty$, $RHS\rightarrow 0$, which means $-Lu_{0}+\lambda u_{0}=0$. Then $u_{0}$ is the solution of equation
\begin{equation}
    \left\{
        \begin{aligned}
            (L-\lambda)u&=0,x\in\Omega\\
            u&=0,x\in\partial\Omega.\\
        \end{aligned}
    \right.
\end{equation}
Contradict! So $\exists C$ such that $\|-Lu+\lambda u\|_{-1}\ge C\|u\|_{1}$ is always true for $u\in C_{c}^{\infty}(\Omega)$.
\end{document}