\documentclass[a4paper]{ctexart}
% useful packages.
\usepackage{amsfonts}
\usepackage{amsmath}
\usepackage{amssymb}
\usepackage{amsthm}
\usepackage{enumerate}
\usepackage{graphicx}
\usepackage{multicol}
\usepackage{fancyhdr}
\usepackage{layout}
\newtheorem{Definition}{\hspace{2em}定义}
\newtheorem{Example}{\hspace{2em}例}
\newtheorem{Thm}{\hspace{2em}定理}
\newtheorem{Lem}{\hspace{2em}引理}
\newtheorem{cor}{\hspace{2em}推论}
% some common command
\newcommand{\dif}{\mathrm{d}}
\newcommand{\avg}[1]{\left\langle #1 \right\rangle}
\newcommand{\difFrac}[2]{\frac{\dif #1}{\dif #2}}
\newcommand{\pdfFrac}[2]{\frac{\partial #1}{\partial #2}}
\newcommand{\OFL}{\mathrm{OFL}}
\newcommand{\UFL}{\mathrm{UFL}}
\newcommand{\fl}{\mathrm{fl}}
\newcommand{\op}{\odot}
\newcommand{\cp}{\cdot}
\newcommand{\Eabs}{E_{\mathrm{abs}}}
\newcommand{\Erel}{E_{\mathrm{rel}}}
\newcommand{\DR}{\mathcal{D}_{\widetilde{LN}}^{n}}
\newcommand\tbbint{{-\mkern -16mu\int}}
\newcommand\tbint{{\mathchar '26\mkern -14mu\int}}
\newcommand\dbbint{{-\mkern -19mu\int}}
\newcommand\dbint{{\mathchar '26\mkern -18mu\int}}
\newcommand\bint{
{\mathchoice{\dbint}{\tbint}{\tbint}{\tbint}}
}
\newcommand\bbint{
{\mathchoice{\dbbint}{\tbbint}{\tbbint}{\tbbint}}
}
\title{Homework\# 1}
\author{Shuang Hu}
\begin{document}
\maketitle
\section*{P86 Problem7}
\begin{proof}
Consider the equation
\begin{equation}
    \left\{
        \begin{aligned}
            \Delta u&=0\text{ in }B(0,r)\\
            u&=g\text{ on }\partial B(0,r),\\
        \end{aligned}
    \right.
\end{equation}
where $g\ge 0$ is always true. Then by Poisson's formula, we can see:
\begin{equation}
    u(x)=\frac{r^{2}-|x|^{2}}{n\alpha(n)r}\int_{\partial B(0,r)}\frac{g(y)}{|x-y|^{n}}\dif S(y).
\end{equation}
The left-hand inequality means:
\begin{equation}
    \begin{aligned}
        &r^{n-2}\frac{r-|x|}{(r+|x|)^{n-1}}u(0)\le\frac{(r^{2}-|x|^{2})}{n\alpha(n)r}\int_{\partial B(0,r)}\frac{g(y)}{|x-y|^{n}}\dif S(y)\\
        \Leftrightarrow&
        n\alpha(n)r^{n-1}u(0)\le(r+|x|)^{n}\int_{\partial B(0,r)}\frac{g(y)}{|x-y|^{n}}\dif S(y)\\
        \Leftrightarrow&
        \int_{\partial B(0,r)}g(y)\dif S(y)\le(r+|x|)^{n}\int_{\partial B(0,r)}\frac{g(y)}{|x-y|^{n}}\dif S(y)\\
    \end{aligned}
\end{equation}
$|y|=r$, and $g(y)\ge 0$, so:$(r+|x|)^{n}\ge|x-y|^{n}$, which means the inequality is true.

In the same way, we can see the right hand is equal to the inequality
\begin{equation}
    (r-|x|)^{n}\int_{\partial B(0,r)}\frac{g(y)}{|x-y|^{n}}\dif S(y)\le \int_{\partial B(0,r)}g(y)\dif S(y).
\end{equation}
As $(r-|x|)^{n}\le|x-y|^{n}$, the inequality is true.
\end{proof}
\section*{P86 Problem8}
\begin{proof}
    (1) As $y\in\partial B(0,r)$, we can see:
    $$
    K(x,y)\in C^{\infty}(B^{0}(0,r))\forall y\in\partial B(0,r).
    $$
    it means that $u(x)\in C^{\infty}(B^{0}(0,r))$.

    (2) As $K(x,y)$ is smooth related to $x$, we can see:
    \begin{equation}
        \begin{aligned}
            \Delta u&=\Delta\int_{\partial B(0,r)}K(x,y)g(y)\dif S(y)\\
            &=\int_{\partial B(0,r)}\Delta_{x}K(x,y)g(y)\dif S(y)\\
            &=0.
        \end{aligned}
    \end{equation}

    (3) Set $g\equiv 1$, we can see:
    \begin{equation}
        \int_{\partial B(0,r)}K(x,y)\dif S(y)=1.
    \end{equation}
    Now fix $x_{0}\in\partial B(0,r)$, $\epsilon>0$, choose $\delta>0$ s.t. $|g(y)-g(x_{0})|<\epsilon$ 
    if $|y-x_{0}|<\delta$. Then:
    \begin{equation}
        \begin{aligned}
            |u(x)-g(x_{0})|&=|\int_{\partial B(0,r)}K(x,y)(g(y)-g(x_{0}))\dif S(y)|\\
            &\le\int_{\partial B(0,r)\cap B(x_{0},\delta)}K(x,y)|g(y)-g(x_{0})|\dif S(y)\\
            &+\int_{\partial B(0,r)\setminus B(x_{0},\delta)}K(x,y)|g(y)-g(x_{0})|\dif S(y)\\
            &<\epsilon+2\|g\|_{\infty}\int_{\partial B(0,r)\setminus B(x_{0},\delta)}K(x,y)\dif S(y)\\
        \end{aligned}
    \end{equation}
    Set $x\rightarrow x_{0}$, we can see $RHS\rightarrow \epsilon$. It means that $u(x)\rightarrow g(x_{0})$ 
    if $x\rightarrow x_{0}$. 
\end{proof}
\section*{P86 Problem9}
\begin{equation}
    u(x)=\frac{2x_{n}}{n\alpha(n)}\int_{\partial\mathbb{R}_{+}^{n}}\frac{g(y)}{|x-y|^{n}}\dif y
    \Rightarrow u(\lambda e_{n})=\frac{2\lambda}{n\alpha(n)}\int_{\partial\mathbb{R}_{+}^{n}}\frac{g(y)}{(\sqrt{\lambda^{2}+|y|^{2}})^{n}}\dif y
\end{equation}
For $u(0)=0$, we can see:
\begin{equation}
\frac{u(\lambda e_{n})-u(0)}{\lambda}=\frac{2}{n\alpha(n)}\int_{\partial\mathbb{R}_{+}^{n}}\frac{g(y)}{(\sqrt{\lambda^{2}+|y|^{2}})^{n}}\dif y
\end{equation}
For $g(y)$ is bounded, we can see the integral
\begin{equation}
    \frac{2}{n\alpha(n)}\int_{|y|\ge1}\frac{g(y)}{(\sqrt{\lambda^{2}+|y|^{2}})^{n}}\dif y
\end{equation}
is convergent. What's more:
\begin{equation}
    \begin{aligned}
        &\frac{2}{n\alpha(n)}\int_{|y|\le1}\frac{g(y)}{(\sqrt{\lambda^{2}+|y|^{2}})^{n}}\dif y\\
        =&\frac{2}{n\alpha(n)}\int_{|y|\le 1}\frac{|y|}{(\sqrt{\lambda^{2}+|y|^{2}})^{n}}\dif y\\
        \rightarrow&+\infty
    \end{aligned}
\end{equation}
when $\lambda\rightarrow 0^{+}$. So we can see $Du$ isn't bounded.
\section*{P86 Problem10}
\begin{proof}
    (a) Just derive the Laplacian of function v, we can see $\Delta v\equiv 0$, which means $v$ is harmonic.

    (b) It suffices to show that the mean value property is true for $x\in\partial U^{+}$.
    $\forall B(x,\delta)\subset U^{+}$, consider the integral
    $$
    I=\int_{\partial B(x,\delta)}v(y)\dif S(y).
    $$
    By the definition, we can see $v(x_{1},\cdots,x_{n})=-v(x_{1},\cdots,-x_{n})$. So we can see that $I=0$, 
    which means the mean value property is true for the function $v$ in $U^{+}$.
\end{proof}
\section*{P87 Problem11}
Let $\Omega\subset\mathbb{R}^n$ be an open subset. If $0\notin\Omega$, we denote $x^*=\frac{x}{|x|^2}$, $\Omega^*=\{x^*|x\in\Omega\}$. For a function $u$ defined on $\Omega$, we define its Kelvin transformation by $K[u](x)=|x|^{2-n}u(x^*)$, $x\in\Omega^*$. Prove that $u$ is harmonic at $\Omega$ if and only if $K[u]$ is harmonic at $\Omega^*$.
\begin{proof}
Note that
\[\frac{\partial|x|}{\partial x_i}=\frac{x_i}{|x|}\]
\[\frac{\partial x_j^*}{\partial x_i}=x_j\cdot(-2)|x|^{-3}\frac{\partial|x|}{\partial x_i}=
-2x_ix_j|x|^{-4}\quad(j\neq i)\]
\[\frac{\partial x_i^*}{\partial x_i}=|x|^{-2}-2x^2_i|x|^{-4}.\]
\begin{align*}
\frac{\partial K[u]}{\partial x_i}
&=(2-n)|x|^{1-n}\frac{\partial|x|}{\partial x_i}u(x^*)+
|x|^{2-n}\sum_{j=1}^{n}u_j(x^*)\frac{\partial x_j^*}{\partial x_i}\\
&=(2-n)|x|^{-n}x_iu(x^*)-2|x|^{-2-n}\sum_{j=1}^{n}u_j(x^*)x_ix_j+|x|^{-n}u_i(x^*).
\end{align*}
\begin{align*}
\frac{\partial^2 K[u]}{\partial x_i^2}
&=n(n-2)|x|^{-n-1}\frac{\partial|x|}{\partial x_i}x_iu(x^*)+
(2-n)|x|^{-n}u(x^*)+(2-n)|x|^{-n}x_i\sum_{j=1}^{n}u_j\frac{\partial x_j^*}{\partial x_i}+\\
&2(n+2)|x|^{-n-3}\frac{\partial|x|}{\partial x_i}\sum_{j=1}^{n}u_jx_ix_j-
2|x|^{-n-2}\sum_{j,k=1}^{n}u_{jk}\frac{\partial x_k^*}{\partial x_i}x_ix_j-
2|x|^{-n-2}\sum_{j=1}^{n}u_jx_j-\\
&2|x|^{-n-2}u_ix_i-n|x|^{-n-1}\frac{\partial|x|}{\partial x_i}u_i+|x|^{-n}\sum_{m=1}^{n}u_{im}
\frac{\partial x_m^*}{\partial x_i}\\
&=n(n-2)|x|^{-n-2}x_i^2u(x^*)+(2-n)|x|^{-n}u(x^*)-\\
&2(2-n)|x|^{-n-4}x_i^2\sum_{j=1}^{n}u_jx_j+(2-n)|x|^{-n-2}u_ix_i+\\
&2(n+2)|x|^{-n-4}x_i^2\sum_{j=1}^{n}u_jx_j+4|x|^{-n-6}\sum_{j,k=1}^{n}u_{jk}x_i^2x_jx_k-\\
&2|x|^{-n-4}\sum_{j=1}^{n}u_{ij}x_ix_j-2|x|^{-n-2}\sum_{j=1}^{n}u_jx_j-\\
&2|x|^{-n-2}u_ix_i-n|x|^{-n-2}u_ix_i+|x|^{-n-2}u_{ii}-2|x|^{-n-4}\sum_{m=1}^{n}u_{im}x_ix_m\\
&=n(n-2)|x|^{-n-2}x_i^2u(x^*)+(2-n)|x|^{-n}u(x^*)+\\
&4n|x|^{-n-4}x_i^2\sum_{j=1}^{n}u_jx_j+4|x|^{-n-6}\sum_{j,k=1}^{n}u_{jk}x_i^2x_jx_k-\\
&2|x|^{-n-4}\sum_{j=1}^{n}u_{ij}x_ix_j-2|x|^{-n-2}\sum_{j=1}^{n}u_jx_j-2n|x|^{-n-2}u_ix_i+\\
&|x|^{-n-2}u_{ii}-2|x|^{-n-4}\sum_{m=1}^{n}u_{im}x_ix_m\\
\end{align*}
\begin{align*}
\Delta K[u]
&=\sum_{i=1}^{n}\frac{\partial^2 K[u]}{\partial x_i^2}\\
&=4|x|^{-n-6}\sum_{i,j,k=1}^{n}u_{jk}x_i^2x_jx_k-2|x|^{-n-4}\sum_{i,j=1}^{n}u_{ij}x_ix_j-
2|x|^{-n-4}\sum_{i,m=1}^{n}u_{im}x_ix_m+|x|^{-n-2}\Delta u\\
&=4|x|^{-n-4}\sum_{j,k=1}^{n}u_{jk}x_jx_k-2|x|^{-n-4}\sum_{i,j=1}^{n}u_{ij}x_ix_j-
2|x|^{-n-4}\sum_{i,m=1}^{n}u_{im}x_ix_m+|x|^{-n-2}\Delta u\\
&=|x|^{-n-2}\Delta u.
\end{align*}
Hence $u$ is harmonic if and only if $K[u]$ is harmonic.
\end{proof}
\end{document}